\documentclass[11pt,twoside]{article}
\usepackage{geometry}
\usepackage{enumerate}
\usepackage{latexsym,booktabs}
\usepackage{amsmath,amssymb,bbold, xfrac}
\usepackage{graphicx}
\usepackage[singlespacing]{setspace}
\usepackage{algorithm}
\usepackage[noend]{algpseudocode}
\usepackage{multirow}

\usepackage{hyperref}
\hypersetup{colorlinks=True, urlcolor=cyan}

\usepackage[backend=bibtex, citestyle=authoryear, bibstyle=alphabetic]{biblatex}
\addbibresource{literature.bib}

\geometry{a4paper,left=2cm,right=2.0cm, top=2cm, bottom=2.0cm}

\newtheorem{Definition}{Definition}
\newtheorem{Theorem}{Theorem}
\newtheorem{Lemma}{Lemma}
\newtheorem{Corollary}{Corollary}
\newtheorem{Proposition}{Proposition}
\newtheorem{Algorithm}{Algorithm}
\numberwithin{Theorem}{section}
\numberwithin{Definition}{section}
\numberwithin{Lemma}{section}
\numberwithin{Algorithm}{section}
\numberwithin{equation}{section}

% Python style code
\usepackage[cache=false]{minted}
\usemintedstyle{tango}
\usepackage{listings}
\usepackage{xcolor}
\newcommand{\pinline}[1]{\mintinline{python}{#1}}
\newminted{python}{framesep=2mm,
baselinestretch=1.2,
fontsize=\small,
frame=single
}



% Define custom commands
\newcommand{\bigO}{\mathcal{O}}
%% Define notation
\newcommand{\X}{\mathbf{x}}
\newcommand{\Z}{\mathbf{z}}
\newcommand{\V}{\mathbf{V}}
\newcommand{\Y}{\mathbf{Y}}
\newcommand{\hess}{\mathbf{H}}
\newcommand{\Thetab}{\mathbf{\Theta}}

\newcommand{\vb}{\mathbf{v}}
\newcommand{\yb}{\mathbf{y}}
\newcommand{\xb}{\mathbf{x}}

\newcommand{\jac}{\mathbf{J}}
\newcommand{\hessian}{\mathbf{H}}
\newcommand{\thetab}{\boldsymbol{\theta}}
\newcommand{\thetabij}{\thetab_{ij}}
\newcommand{\thetabi}{\thetab_i}

\newcommand{\region}{B_{d,\epsilon}}
\newcommand{\indicator}[1]{\mathbb{1}_{#1}}
\newcommand{\regioni}{B_{d,\epsilon}^i}
\newcommand{\data}{\mathbf{y_0}}

\newcommand{\R}{\mathbb{R}}


% \newcommand{\likelihood}{L(\theta)}
% \newcommand{\loglikelihood}{l(\theta)}
% \newcommand{\likelihoodabc}{L_{d, \epsilon}}
% \newcommand{\posterior}{p(\thetab|\Xobs)}
% \newcommand{\posteriorabc}{p_{\epsilon}(\theta|\Xobs)}
% \newcommand{\nuisance}{\mathbf{u}}
% \newcommand{\model}{\gen(\thetab,\nuisance)}
% \newcommand{\xobs}{x^0}
% \newcommand{\Xobs}{\X^0}
% \newcommand{\xsim}{x_i}
% \newcommand{\Xsim}{\X_i}
% \newcommand{\region}{C_\epsilon(\nuisance)}
% \newcommand{\regionpair}{C_\epsilon}

% \newcommand{\regionuni}{U^i_\epsilon(\thetab)}
% \newcommand{\optend}{\thetab^*_i}
% \newcommand{\Doptend}{d^*_i}

% \newcommand{\optendname}{\text{optimisation end point}}   % In-text name for what I call "OMC sample"

% \newcommand{\dete}{|\det (\A_i)|}
% \newcommand{\determ}{\dete^{-1}}

% \newcommand{\parameters}{\theta}
% \newcommand{\gpmodel}{\hat{d}_i}
% \newcommand{\image}{x(\theta, \nuisance)}
% \newcommand{\distance}{d(\image, \xobs)}
% \newcommand{\gradfull}{\frac{\partial \distance}{\partial \parameters}}
% \newcommand{\gradopendr}{\frac{\partial \image}{\partial \parameters}}
% \newcommand{\gradrecog}{\frac{\partial \distance}{\partial \image}}
% \newcommand{\recog}{r(x)}
% \newcommand{\thetau}{\theta_i(u_j)}


% \newcommand{\sumstat}{\Phi}
% \newcommand{\gensum}{\mathbf{f}}
% \newcommand{\sumstatObs}{\mathbf{y}^0}
% \newcommand{\E}{\mathop{\mathbb{E}}}
% \newcommand{\M}{\mathbf{M}}
% \newcommand{\A}{\mathbf{A}}
% \newcommand{\gen}{g}
% \newcommand{\voli}{\text{vol}(\regioni)}
% \newcommand{\vole}{\text{vol}(B_{\epsilon})}
% \newcommand{\w}{\mathbf{w}}
% \newcommand{\hpost}{\bar{h}} % \E[h(\thetab) | \Xobs]
% \newcommand{\ESS}{\textrm{ESS}}

% %% Define transpose symbol
% \newcommand{\tran}{\mkern-2mu\raise1.25ex\hbox{$\scriptscriptstyle\top$}\mkern-3.5mu}
% \newcommand{\tranlow}{\mkern-2mu\raise0.75ex\hbox{$\scriptscriptstyle\top$}\mkern-3.5mu}

% %% for formatting of paragraphs in Section 3 that describe the paragraphs
% \newcommand{\algformat}[1]{\textbf{#1}}
% %\newcommand{\algformat}[1]{\emph{#1}}
% %\newcommand{\algformat}[1]{#1}

\begin{document}

\pagestyle{empty}

% =============================================================================
% Title page
% =============================================================================
\begin{titlepage}
\vspace*{.5em}
\center
\textbf{\large{The School of Mathematics}} \\
\vspace*{1em}
\begin{figure}[!h]
\centering
\includegraphics[width=180pt]{Thesis/images/CentredLogoCMYK.jpg}
\end{figure}
\vspace{2em}
\textbf{\Huge{Robust Optimisation Monte Carlo for Likelihood-Free Inference}}\\[2em]
\textbf{\LARGE{by}}\\
\vspace{2em}
\textbf{\LARGE{Vasileios Gkolemis}}\\
\vspace{6.5em}
\Large{Dissertation Presented for the Degree of\\
MSc in Operational Research with Data Science}\\
\vspace{6.5em}
\Large{August 2020}\\
\vspace{3em}
\Large{Supervised by\\Dr. Michael Gutmann}
\vfill
\end{titlepage}

\cleardoublepage
n
% =============================================================================
% Abstract, acknowledgments, and own work declaration
% =============================================================================
\begin{center}
\Large{Abstract}
\end{center}


\clearpage

\begin{center}
\Large{Acknowledgments}
\end{center}


\clearpage

\begin{center}
\Large{Own Work Declaration}
\end{center}

Here comes your own work declaration

\cleardoublepage



% =============================================================================
% Table of contents, tables, and pictures (if applicable)
% =============================================================================
\pagestyle{plain}
\setcounter{page}{1}
\pagenumbering{Roman}

\tableofcontents
\clearpage
\listoftables
\listoffigures
\cleardoublepage

\pagenumbering{arabic}
\setcounter{page}{1}

\nocite{*}

%%%%%%%%%%%%%%%%%%%%%%%%%%%%%%%%%%%%%%%%
\clearpage
\section{Introduction}
\label{sec:introduction}
This dissertation is mainly focused on the implementation of the Robust Optimisation Monte Carlo (ROMC) method as it was proposed by \autocite{Ikonomov2019} at the python package ELFI (Engine For Likelihood-Free Inference) \autocite{1708.00707}. The ROMC method describes a novel likelihood-free inference approach for simulator-based models.

\subsection{Motivation}
\label{subsec:motivation}
% \subsubsection*{\textit{Explanation of simulation-based models}}

A simulator-based model is a parameterised stochastic data generating
mechanism \autocite{Gutmann2016}. The key characteristic of these models is that although we can sample (simulate) data points, we cannot
evaluate the likelihood for a specific set of observations
$\data$. Formally, a simulator-based model is described as a
parameterised family of probability density functions
$\{ p_{\yb|\thetab}(\yb) \}_{\thetab}$, whose closed-form is either
unknown or intractable to evaluate. Whereas evaluating
$p_{\yb|\thetab}(\yb)$ is intractable, sampling is feasible. Practically, a
simulator can be understood as a black-box machine $M_r$\footnote{The
  subscript $r$ in $M_r$ indicates the \textit{random} simulator. In
  the next chapters we will introduce $M_d$ witch stands for the
  \textit{deterministic} simulator.} that given a set of parameters $\thetab$,
produces samples $\yb$ in a stochastic manner i.e.\
$M_r(\thetab) \rightarrow \yb$.

Simulator-based models are particularly captivating due to the
modelling freedom they provide; any physical
process that can be conceptualised as a computer program of finite
(deterministic or stochastic) steps can be modelled as a
simulator-based model with any more mathematical compromise. This includes any amount of hidden (unobserved) internal variables or
logic-based decisions. As always, this degree of freedom comes
at a cost; performing the inference is particularly demanding from both the 
computational and the mathematical perspective. Unfortunately, the
algorithms deployed so far, permit the inference only at
low-dimensionality parametric spaces, i.e.\ $\thetab \in \mathbb{R}^D$
where $D$ is small.

\subsubsection*{\textit{Example}}

For illustrating the importance of simulator-based models, let us use
the tuberculosis disease spread example as described in
\autocite{Tanaka2006}. An overview of the disease spread model is
presented at figure~\ref{fig:tuberculosis_model}. At each stage one of
the following \textit{unobserved} events may happen; (a) the
transmission of a specific haplotype to a new host (b) the mutation of
an existent haplotype (c) the exclusion of an infectious host
(recovers/dies) from the population. The random process, which stops
when $m$ infectious hosts are reached\footnote{We suppose that the
  unaffected population is infinite, so a new host can always be added
  until we reach $m$ simultaneous hosts.}, can be parameterised (a) by
the transmission rate $\alpha$ (b) the mutation rate $\tau$ and (c)
the exclusion rate $\delta$, creating a $3D$-parametric space
$\thetab = (\alpha, \tau, \delta)$. The outcome of the process is a
variable-size tuple $\yb_{\thetab}$, containing the population
contaminated by each different haplotype, as described in
figure~\ref{fig:tuberculosis_model}. Lets say that the disease has
been spread in a real population and when $m$ hosts where contaminated
simultaneously, the vector with the infectious populations has been
measured to be $\data$. We would like to discover the parameters
$\thetab = (\alpha, \tau, \delta)$ that describe the spreading process
and lead to the specific outcome $\data$. Computing
$p(\yb=\data|\thetab)$ requires tracking all tree-paths that generate
the specific tuple along with their probabilities and summing over
them. Computing this probability by enumerating each possible
tree-path that may lead to the specific outcome becomes intractable
when $m$ grows larger, as in real-case scenarios. This can be easily
observed in the tree presented at
figure~\ref{fig:tuberculosis_model}. On the other hand, modelling the
data-generation process as a computer program is simple and
computationally efficient, hence using a simulator-based Model is a
perfect fit.

\begin{figure}[!ht]
    \begin{center}
      \includegraphics[width=0.75\textwidth]{./Thesis/images/chapter1/tuber_model_1.png}
    \end{center}
    \caption{Depiction of a spread outcome of the tuberculosis spreading process. The image has been taken from \autocite{Lintusaari2017}}
    \label{fig:tuberculosis_model}
\end{figure}

\subsubsection*{\textit{Goal of Simulation-Based Models}}

As in most Machine Learning (ML) concepts, the fundamental goal is the
derivation of one(many) parameter configuration(s) $\thetab^*$ that
\textit{describe} well the data i.e.\ generate samples
$M_r(\thetab^*)$ that are as close as possible to the observed data
$\data$. In our case, following the approach of Bayesian Machine
Learning, we treat the parameters of interest $\thetab$ as random
variables and we try to \textit{infer} a posterior distribution
$p(\thetab|\data)$ on them. 

\subsubsection*{\textit{Robust Optimisation Monte Carlo (ROMC) method}}

The ROMC method~\autocite{Ikonomov2019} is very a recent
Likelihood-Free approach; its fundamental idea is the transformation
of the stochastic data generation process $M_r(\thetab)$ to a
deterministic mapping $g_i(\thetab)$, by sampling the variables that
produce the randomness $\vb_i \sim p(\V)$. Formally, in every
stochastic process the randomness is influenced by a vector of random
variables $\V$, whose state is unknown before the execution of the
simulation; sampling the state makes the procedure deterministic,
namely $g_i(\thetab) = M_d(\thetab, \V=\vb_i)$. This approach
initially introduced at \autocite{Meeds2015} with the title
\textit{Optimisation Monte Carlo (OMC)}. The ROMC extended this
approach by resolving a fundamental failure-mode of OMC\@. The ROMC
describes a methodology for approximating the posterior through a
series of algorithmic steps, without explicitly enforcing which
algorithms must be utilised for each step\footnote{The implementation
  chooses a specific algorithm for each task, but this choice has just
  a demonstrative value; any appropriate algorithm can be used
  instead.}; in this sense, it can be thought as a meta-algorithm.

\subsubsection*{\textit{Implementation}}

The most important contribution of this work is the implementation of
the ROMC method in the Python package Engine for Likelihood-Free
Inference (ELFI) \autocite{1708.00707}. Since the method by published quite recently, if has not been implemented until now in any ML software. This work attempts to provide to the research community a
tested and robust implementation for further experimentation and
possible extensions.

\subsubsection*{\textit{Explanation of simulation-based models}}

A simulator-based model is a parameterised stochastic data generating
mechanism \autocite{Gutmann2016}. The key characteristic of these models is that although we can sample (simulate) data points, we cannot
evaluate the likelihood for a specific set of observations
$\data$. Formally, a simulator-based model is described as a
parameterised family of probability density functions
$\{ p_{\yb|\thetab}(\yb) \}_{\thetab}$, whose closed-form is either
unknown or intractable to evaluate. Whereas evaluating
$p_{\yb|\thetab}(\yb)$ is intractable, sampling is feasible. Practically, a
simulator can be understood as a black-box machine $M_r$\footnote{The
  subscript $r$ in $M_r$ indicates the \textit{random} simulator. In
  the next chapters we will introduce $M_d$ witch stands for the
  \textit{deterministic} simulator.} that given a set of parameters $\thetab$,
produces samples $\yb$ in a stochastic manner i.e.\
$M_r(\thetab) \rightarrow \yb$.

Simulator-based models are particularly captivating due to the
modelling freedom they provide; any physical
process that can be conceptualised as a computer program of finite
(deterministic or stochastic) steps can be modelled as a
simulator-based model with any more mathematical compromise. This includes any amount of hidden (unobserved) internal variables or
logic-based decisions. As always, this degree of freedom comes
at a cost; performing the inference is particularly demanding from both the 
computational and the mathematical perspective. Unfortunately, the
algorithms deployed so far, permit the inference only at
low-dimensionality parametric spaces, i.e.\ $\thetab \in \mathbb{R}^D$
where $D$ is small.

\subsubsection*{\textit{Example}}

For illustrating the importance of simulator-based models, let us use
the tuberculosis disease spread example as described in
\autocite{Tanaka2006}. An overview of the disease spread model is
presented at figure~\ref{fig:tuberculosis_model}. At each stage one of
the following \textit{unobserved} events may happen; (a) the
transmission of a specific haplotype to a new host (b) the mutation of
an existent haplotype (c) the exclusion of an infectious host
(recovers/dies) from the population. The random process, which stops
when $m$ infectious hosts are reached\footnote{We suppose that the
  unaffected population is infinite, so a new host can always be added
  until we reach $m$ simultaneous hosts.}, can be parameterised (a) by
the transmission rate $\alpha$ (b) the mutation rate $\tau$ and (c)
the exclusion rate $\delta$, creating a $3D$-parametric space
$\thetab = (\alpha, \tau, \delta)$. The outcome of the process is a
variable-size tuple $\yb_{\thetab}$, containing the population
contaminated by each different haplotype, as described in
figure~\ref{fig:tuberculosis_model}. Lets say that the disease has
been spread in a real population and when $m$ hosts where contaminated
simultaneously, the vector with the infectious populations has been
measured to be $\data$. We would like to discover the parameters
$\thetab = (\alpha, \tau, \delta)$ that describe the spreading process
and lead to the specific outcome $\data$. Computing
$p(\yb=\data|\thetab)$ requires tracking all tree-paths that generate
the specific tuple along with their probabilities and summing over
them. Computing this probability by enumerating each possible
tree-path that may lead to the specific outcome becomes intractable
when $m$ grows larger, as in real-case scenarios. This can be easily
observed in the tree presented at
figure~\ref{fig:tuberculosis_model}. On the other hand, modelling the
data-generation process as a computer program is simple and
computationally efficient, hence using a simulator-based Model is a
perfect fit.

\begin{figure}[!ht]
    \begin{center}
      \includegraphics[width=0.75\textwidth]{./Thesis/images/chapter1/tuber_model_1.png}
    \end{center}
    \caption{Depiction of a spread outcome of the tuberculosis spreading process. The image has been taken from \autocite{Lintusaari2017}}
    \label{fig:tuberculosis_model}
\end{figure}

\subsubsection*{\textit{Goal of Simulation-Based Models}}

As in most Machine Learning (ML) concepts, the fundamental goal is the
derivation of one(many) parameter configuration(s) $\thetab^*$ that
\textit{describe} well the data i.e.\ generate samples
$M_r(\thetab^*)$ that are as close as possible to the observed data
$\data$. In our case, following the approach of Bayesian Machine
Learning, we treat the parameters of interest $\thetab$ as random
variables and we try to \textit{infer} a posterior distribution
$p(\thetab|\data)$ on them. 

\subsubsection*{\textit{Robust Optimisation Monte Carlo (ROMC) method}}

The ROMC method~\autocite{Ikonomov2019} is very a recent
Likelihood-Free approach; its fundamental idea is the transformation
of the stochastic data generation process $M_r(\thetab)$ to a
deterministic mapping $g_i(\thetab)$, by sampling the variables that
produce the randomness $\vb_i \sim p(\V)$. Formally, in every
stochastic process the randomness is influenced by a vector of random
variables $\V$, whose state is unknown before the execution of the
simulation; sampling the state makes the procedure deterministic,
namely $g_i(\thetab) = M_d(\thetab, \V=\vb_i)$. This approach
initially introduced at \autocite{Meeds2015} with the title
\textit{Optimisation Monte Carlo (OMC)}. The ROMC extended this
approach by resolving a fundamental failure-mode of OMC\@. The ROMC
describes a methodology for approximating the posterior through a
series of algorithmic steps, without explicitly enforcing which
algorithms must be utilised for each step\footnote{The implementation
  chooses a specific algorithm for each task, but this choice has just
  a demonstrative value; any appropriate algorithm can be used
  instead.}; in this sense, it can be thought as a meta-algorithm.

\subsubsection*{\textit{Implementation}}

The most important contribution of this work is the implementation of
the ROMC method in the Python package Engine for Likelihood-Free
Inference (ELFI) \autocite{1708.00707}. Since the method by published quite recently, if has not been implemented until now in any ML software. This work attempts to provide to the research community a
tested and robust implementation for further experimentation and
possible extensions.


\subsection{Outline of Thesis}
\label{subsec:outline-of-thesis}
% The remainder of the dissertation is organized as follows. In Chapter
2 we establish the mathematical formulation; more specifically we
initially describe the Simulator-Based models and we provide some
fundamental algorithms that have been proposed for performing
statistical inference. Afterwards, we provide the mathematical
description of the ROMC approach \autocite{Ikonomov2019}. In Chapter 3, we
deal with the implementation part; we initially provide some
information regarding the Python package Engine for Likelihood-Free
Inference (ELFI) \autocite{1708.00707} and subsequently we present the
implementation details of ROMC in this package. In Chapter 4, we
illustrate the functionalities of the ROMC implementation at some
real-world examples; this chapter wishes to demonstrate the success of
the ROMC method and of our implementation at Likelihood-Free
tasks. Finally, in Chapter 5, we conclude with some thoughts on the
work we have done and some future research ideas.

The remainder of the dissertation is organized as follows. In Chapter
2 we establish the mathematical formulation; more specifically we
initially describe the Simulator-Based models and we provide some
fundamental algorithms that have been proposed for performing
statistical inference. Afterwards, we provide the mathematical
description of the ROMC approach \autocite{Ikonomov2019}. In Chapter 3, we
deal with the implementation part; we initially provide some
information regarding the Python package Engine for Likelihood-Free
Inference (ELFI) \autocite{1708.00707} and subsequently we present the
implementation details of ROMC in this package. In Chapter 4, we
illustrate the functionalities of the ROMC implementation at some
real-world examples; this chapter wishes to demonstrate the success of
the ROMC method and of our implementation at Likelihood-Free
tasks. Finally, in Chapter 5, we conclude with some thoughts on the
work we have done and some future research ideas.


\subsection{Notation}
\label{subsec:notation}
% \subsection{Notation}
\label{subsec:notation}

In this section, we keep an overview of the symbols used throughout
the document along with their explanation, in physical language. Some
quantities that are discribed informally here will be defined formally
in the next chapters. We try to keep the notation as consistent as
possible throughout the document.

 Some more generic symbols. In general, as $p(\cdot)$ we represent any valid pdf and $p(\cdot | \cdot)$ any conditional distribution. Depending on the content and the symbol used at the placeholder the distribution may have a more specific meaning.


\subsubsection*{Random Generator:}
\label{sec:random-generator}
\begin{itemize}
\item $M_r(\thetab): \R^D \rightarrow \R$
\end{itemize}

\subsubsection*{Variables}
\label{sec:variables}

\begin{itemize}
\item $D$: the dimensionality of the parameter-space
\item $\thetab \in \R^D$: the parameters of interest
\item $\data$: the observations
\item $\epsilon$: threshold
\item $\vb \in \R^N$, random variable that adds the stochasticity to the generator. 
\item $\vb_i \sim \vb$: a sample drawn from $\vb$
\item $\Y_{\thetab}$ random variable descriping the simulator $M_r(\thetab)$. The pdf of $\Y_{\thetab}$ is unknown in closed form or intractable to be evaluated.
\item $\yb_i \sim \Y_\theta$ a sample drawn from $\Y_\theta$. The sample can be obtained by executing the simulator $\yb_i \sim M_r(\thetab)$
\end{itemize}


\subsubsection*{Sets}
\label{sec:sets}

\begin{itemize}
\item $\region(\data)$ the set of points $\yb$ around the observations $\data$, i.e. $\yb := \{\yb: d(\yb, \data) < \epsilon \}$
\item $\regioni = \region(\yb_i)$ the set of points $\yb$ around $\yb_i$, i.e. $\yb := \{\yb: d(\yb, \yb_i) < \epsilon \}$
\end{itemize}
    
\subsubsection*{Generic Functions}
\label{sec:generic-functions}

\begin{itemize}
\item $p(\thetab)$: the prior distribution
\item $p(\thetab|\data)$: the posterior distribution
\item $p_{d,\epsilon}(\thetab|\data)$: the approximate posterior distribution
\item $d(\mathbf{x}, \mathbf{y}): \R^{2N} \rightarrow \R$: any valid distance, e.g L2 norm: $||\mathbf{x}-\mathbf{y}||_2^2$
\end{itemize}

\subsubsection*{Functions (Mappings):}
\label{sec:functions-mappings}

\begin{itemize}
\item $M_d(\thetab, \vb): \R^D \rightarrow \R$ the deterministic generator; if we pass the state $v$ of all stochastic variables that are part of the data generation process, then producing an outcome becomes deterministic.
\item $f_i(\thetab) = M_d(\thetab, \vb_i)$ alias

\item $g_i(\thetab) = d(f_i(\thetab), \data)$
\item $T(\mathbf{x}): \mathbb{R}^{D_1} \rightarrow \mathbb{R}^{D_2}$ where $D_1 > D_2$, the summary statistic mapping.
\item  $\indicator{\region(\data)}(\yb)$ the indicator function:
    \begin{gather*}\indicator{\region(\data)}(\yb) = \left\{
	\begin{array}{ll}
		1 & \mbox{if } d(\yb,\data) \leq \epsilon \\
		0 & \mbox{else } 
	\end{array} \right. \end{gather*}

\item $L(\theta)$ the likelihood function
\item $L_{d,\epsilon}(\theta)$ the approximate likelihood function
\end{itemize}   

\subsection{Notation}
\label{subsec:notation}

In this section, we keep an overview of the symbols used throughout
the document along with their explanation, in physical language. Some
quantities that are discribed informally here will be defined formally
in the next chapters. We try to keep the notation as consistent as
possible throughout the document.

 Some more generic symbols. In general, as $p(\cdot)$ we represent any valid pdf and $p(\cdot | \cdot)$ any conditional distribution. Depending on the content and the symbol used at the placeholder the distribution may have a more specific meaning.


\subsubsection*{Random Generator:}
\label{sec:random-generator}
\begin{itemize}
\item $M_r(\thetab): \R^D \rightarrow \R$
\end{itemize}

\subsubsection*{Variables}
\label{sec:variables}

\begin{itemize}
\item $D$: the dimensionality of the parameter-space
\item $\thetab \in \R^D$: the parameters of interest
\item $\data$: the observations
\item $\epsilon$: threshold
\item $\vb \in \R^N$, random variable that adds the stochasticity to the generator. 
\item $\vb_i \sim \vb$: a sample drawn from $\vb$
\item $\Y_{\thetab}$ random variable descriping the simulator $M_r(\thetab)$. The pdf of $\Y_{\thetab}$ is unknown in closed form or intractable to be evaluated.
\item $\yb_i \sim \Y_\theta$ a sample drawn from $\Y_\theta$. The sample can be obtained by executing the simulator $\yb_i \sim M_r(\thetab)$
\end{itemize}


\subsubsection*{Sets}
\label{sec:sets}

\begin{itemize}
\item $\region(\data)$ the set of points $\yb$ around the observations $\data$, i.e. $\yb := \{\yb: d(\yb, \data) < \epsilon \}$
\item $\regioni = \region(\yb_i)$ the set of points $\yb$ around $\yb_i$, i.e. $\yb := \{\yb: d(\yb, \yb_i) < \epsilon \}$
\end{itemize}
    
\subsubsection*{Generic Functions}
\label{sec:generic-functions}

\begin{itemize}
\item $p(\thetab)$: the prior distribution
\item $p(\thetab|\data)$: the posterior distribution
\item $p_{d,\epsilon}(\thetab|\data)$: the approximate posterior distribution
\item $d(\mathbf{x}, \mathbf{y}): \R^{2N} \rightarrow \R$: any valid distance, e.g L2 norm: $||\mathbf{x}-\mathbf{y}||_2^2$
\end{itemize}

\subsubsection*{Functions (Mappings):}
\label{sec:functions-mappings}

\begin{itemize}
\item $M_d(\thetab, \vb): \R^D \rightarrow \R$ the deterministic generator; if we pass the state $v$ of all stochastic variables that are part of the data generation process, then producing an outcome becomes deterministic.
\item $f_i(\thetab) = M_d(\thetab, \vb_i)$ alias

\item $g_i(\thetab) = d(f_i(\thetab), \data)$
\item $T(\mathbf{x}): \mathbb{R}^{D_1} \rightarrow \mathbb{R}^{D_2}$ where $D_1 > D_2$, the summary statistic mapping.
\item  $\indicator{\region(\data)}(\yb)$ the indicator function:
    \begin{gather*}\indicator{\region(\data)}(\yb) = \left\{
	\begin{array}{ll}
		1 & \mbox{if } d(\yb,\data) \leq \epsilon \\
		0 & \mbox{else } 
	\end{array} \right. \end{gather*}

\item $L(\theta)$ the likelihood function
\item $L_{d,\epsilon}(\theta)$ the approximate likelihood function
\end{itemize}   


%%%%%%%%%%%%%%%%%%%%%%%%%%%%%%%%%%%%%%%%
\clearpage
\section{Background}
\label{sec:background}

\subsection{Simulator-based models}
% \subsection{Simulator-Based (Implicit) Models}

As already stated, in Simulator-Based models it is impossible to evaluate the quantity $p_{y|\theta}(y)$. The only tool we own is a black-box simulator $M_r(\theta)$ that can be used to generate data. If we denote as $Y_\theta$ the random variable that describes the simulator, then

\begin{equation} 
  Pr(Y_\theta \in B_\epsilon(y_o)) = Pr(M_r(\theta) \in B_\epsilon(y_o)) = \int_{y \in B_\epsilon(y_0)} p(y|\theta)dy
  \end{equation}

  On the other hand, the likelihood function can be defined as:
\begin{equation} \label{eq:likelihood}
  L(\theta) =  \lim_{\epsilon \to 0} c_\epsilon Pr(Y_\theta \in B_\epsilon(y_0)) = \lim_{\epsilon \to 0} c_\epsilon \int_{y \in B_\epsilon(y_0)} p(y|\theta)dy
\end{equation}

and the posterior distribution as:
\begin{equation}
p(\theta|y_0) \propto L(\theta)p(\theta)
\end{equation}

Since $p_{y|\theta}$ cannot be evaluated, so does $L(\theta)$ and subsequently $p(\theta|y_0)$.

\subsubsection{Approximate Bayesian Computation (ABC) Rejection Sampling}

ABC Rejection Sampling is a modified version of Rejection Sampling, for cases when likelihood evaluation is intractable. In the Rejection  method a sample is obtained from the prior $\theta \sim p(\theta)$ and it is maintained with probability $L(\theta)/\text{max}_\theta L(\theta)$. The samples $\theta_i$ obtained with this procedure follow the posterior distribution $p(\theta|y_0)$. Although we cannot use this approach out of the box (evaluating $L(\theta)$ is impossible in our case), we can adjust it with some slight modifications.

In the discrete case scenario where $Y_\theta$ can take a finite set of numbers, the likelihood becomes $L(\theta) = Pr(Y_\theta=y_0)$ and the posterior $p(\theta|y_o) \propto Pr(Y_\theta=y_o)p(\theta)$. Hence, we can sample from the prior $\theta_i \sim p(\theta)$, run the simulator $y_i = M(\theta_i, V)$ and maintain $\theta_i$ if only $y_i = y_0$.

The above method becomes less usefull as the finite set of possible $Y_\theta$ values grows large. As the set grows larger, the probability to maintain a sample becomes smaller. In the limit where the set becomes infinite (i.e. continuous case) the probability becomes zero. In order for the method to work in this set-up, a relaxation is introduced; we relax the acceptance criterion by letting $Y_\theta$ lie in a contiguous area around $y_0$, i.e. $Y_\theta \in B_\epsilon(y_0), \epsilon > 0$. The area can be defined as $B_\epsilon(y_0) := \{y: d(y, y_0) < \epsilon \}$ where $d(\cdot, \cdot)$ can represent any valid distance. With this modification, the maintained samples follow an approximate posterior

\begin{equation} \label{eq:approx_posterior}
  p_{d,\epsilon}(\theta|y_0) \propto Pr(Y_\theta \in B_\epsilon(y_0))p(\theta)
  \end{equation}

  where $B_\epsilon$ is defined by $d, \epsilon$. This procedure is called Rejection ABC algorithm and forms the basis of Likelihood-Free methods.

\subsubsection{Summary Statistics}

When the dimensionality of $Y_\theta \in \mathbb{R}^D$ is high, generating samples inside $B_\epsilon(y_0)$ becomes rare; this is the curse of dimensionality. As an representative example if $B_\epsilon(y_0) := \{ y: ||y - y_0||_2^2 < \epsilon^2 \}$ is a hyper-sphere with radius $\epsilon$ and the prior distribution $p(\theta)$ is a uniform distribution in a hyper-cube with side of length $2/epsilon$, the probability of drawing a sample inside the hyper-sphere becomes:

\begin{equation}
  Pr(Y_\theta \in B_\epsilon(y_0)) = Pr(\theta \in B_\epsilon(y_0)) = \frac{V_{hypersphere}}{V_{hypercube}} = \frac{\pi^{D/2}}{D2^{D-1}\Gamma(D/2)} \rightarrow 0 \text{as} D \rightarrow \infty
\end{equation}

We observe that the probality tends to $0$, independently of $\epsilon$; enlarging $\epsilon$ will not increase the acceptance rate. This produces the need for a mapping $T: \mathbb{R}^{D_1} \rightarrow \mathbb{R}^{D_2}$ where $D_1 > D_2$, redefining the area as $B_\epsilon(y_0) := \{y: d(T(y), T(y_0)) < \epsilon \}$. This process is called measuring the distance at the \textit{summary statistics} level.

\subsubsection{Approximation}

Approximating the posterior as $p(\theta|y_0) \approx p_{d,\epsilon}(\theta|y_0) \propto Pr(Y_\theta \in B_\epsilon(y_0))p(\theta)$ where $B_\epsilon(y_0) := \{y: d(T(y), T(y_0)) < \epsilon \}$ introduces two approximation errors:

\begin{itemize}
\item $\epsilon$ is chosen to be large enough for enough samples to be accepted
  \item Summary Statistics (a) make the distance not a metric in a formal sense, i.e. $d = 0$, even if $y \neq y_0$ (b) make possible disjoint sets of $y$ to lie inside $B_\epsilon{y_0}$
  \end{itemize}

  In the following sections we will not use the summary statistics in our expression, for the notation not to clutter. We can state that all the following statements are valid with incorporating summary statistics.
  
  \subsubsection{Optimization Monte Carlo (OMC)}

  We have already defined $B_{d,\epsilon}(y) := \{x: d(y,x)<\epsilon\}$ as the set of points that lie inside area defined by $(d, \epsilon, y)$. Based on that, we can define two useful entities; an indicator function and a conditional distribution.

  \subsubsection*{Indicator Function}

The indicator function $\mathbb{1}_{B_{d,\epsilon}(y)}(x)$ returns 1 if $x \in B_{d,\epsilon}(y)$ and 0 otherwise. If $d(\cdot,\cdot)$ is a formal distance, due to symmetry $\mathbb{1}_{B_{d,\epsilon}(y)}(x) = \mathbb{1}_{B_{d,\epsilon}(x)}(y)$.

\begin{gather} \label{eq:indicator}
  \mathbb{1}_{B_{d,\epsilon}(y)}(x) = \left\{
	\begin{array}{ll}
		1 & \mbox{if } x \in B_{d,\epsilon}(y) \\
		0 & \mbox{else } 
	\end{array} \right. \end{gather}

\subsubsection*{Boxcar Kernel}

The boxcar kernel is the conditional distribution:

\begin{gather}
  p_{d,\epsilon}(y|x) = \left\{
	\begin{array}{ll}
		c  & \mbox{if } d(y,x) \leq \epsilon \\
		0 & \mbox{else } 
	\end{array}
  \right. \text{where} c = \frac{1}{\int_{ \{ y: d(y,x) < \epsilon \} } dy}
\end{gather}
%
If we understand the boxcar kernel as a data generation process we can make two important notices:

\begin{itemize}
\item given a specific $x$, all values $y: y \in B_{d,\epsilon}(x)$ have equal probability to be generated
  \item if a specific $y$ value has been generated, all $x: x \in B_{d,\epsilon}(y)$ have equal probability to be the conditional value that lead to this generation
  \end{itemize}
%
Finally, we can also observe that the kernel can be defined through the indicator function:

\begin{equation}
  p_{d,\epsilon}(y|x) = c \mathbb{1}_{B_{d,\epsilon}(y)}(x) = c \mathbb{1}_{B_{d,\epsilon}(x)}(y)
\end{equation}

\subsubsection*{Initial View}

Based on the knowledge we have so far, we could define and approximate the approximate likelihood $L_{d,\epsilon}(\theta)$ and through~\ref{eq:approx_posterior} the approximate posterior $p_{d, \epsilon}(\theta|y_0)$. The approximation is given below:

\begin{gather} \label{eq:primal_view}
  L_{d, \epsilon}(\theta)=\int_{B_\epsilon(y_0)}p(y|\theta)dy = \int p_{d,\epsilon}(y_0|y)p(y|\theta)dy\\
  \approx \frac{1}{N} \sum_i^N p_{d,\epsilon} (y_0|y_i) \\
  \approx \frac{c}{N} \sum_i^N \mathbb{1}_{B_{d,\epsilon}(y_i)} (y_0), y_i \sim M_r(\theta)
\end{gather}
%
This approach is quite intuitive; approximating likelihood of a specific $\theta$ requires sampling from the data generator and count the fraction of samples that lie inside the area around the observed data. On the other hand, it has a major disadvantage; evaluating $L_{d,\epsilon}(\theta)$ requires resampling $N$ points and checking which ones are close to the observed data $y_0$.

\subsubsection*{Alternative View}

OMC attempts an alternative view to the approximation of $L_{d,\epsilon}(\theta)$, which has some advantages. Instead of incorporating the random generator $M_r(\theta)$, it samples all the nuisance variables from a prior distribution $v_i \sim p(v)$ and then it uses the deterministic mapping $M_d(\theta, v_i)$. More formally the approach is the following:

\begin{gather} 
  L_{d,\epsilon}(\theta)=\int_{B_\epsilon(y_0)}p(y|\theta)dy = \int p_{d,\epsilon}(y_0|y)p(y|\theta)dy\\
  = \int_y \int_v p_{d,\epsilon}(y_0|y)p(y|\theta, v) p(v)dxdv \\
  = \int_v p_{d,\epsilon}(y_0|y=M_d(\theta, v)) p(v)dv \\
  \approx \frac{1}{N} \sum_i^N p_{d,\epsilon} (y_0|y=M_d(\theta, v_i)) \\
  \approx \frac{c}{N} \sum_i^N \mathbb{1}_{B_{d,\epsilon}(M_d(\theta, v_i))} (y_0), v_i \sim p(v)
  \label{eq:alt_view}
\end{gather}
%
Based on this approach, the unnormalized approximate posterior can be defined as:

\begin{equation} \label{eq:posterior}
  p_{d,\epsilon}(\theta|y_0) \propto p(\theta) \sum_i^N \mathbb{1}_{B_{d,\epsilon}(M_d(\theta, v_i))} (y_0)
  \end{equation}
%
Forming an analogy with the previous approach, we sample many nuisance variables in order to absorb the randomness of the generator and we count the fraction of times the deterministic generator produces mapps to outputs close to the observed data. Though it is conceptually close to the previous approach, this approach has a major advantage; we can sample the nuisance variables once (training part) and afterwards evaluate every $\theta$ based on a predefined expression (inference part).

\subsection{Simulator-Based (Implicit) Models}

As already stated, in Simulator-Based models it is impossible to evaluate the quantity $p_{y|\theta}(y)$. The only tool we own is a black-box simulator $M_r(\theta)$ that can be used to generate data. If we denote as $Y_\theta$ the random variable that describes the simulator, then

\begin{equation} 
  Pr(Y_\theta \in B_\epsilon(y_o)) = Pr(M_r(\theta) \in B_\epsilon(y_o)) = \int_{y \in B_\epsilon(y_0)} p(y|\theta)dy
  \end{equation}

  On the other hand, the likelihood function can be defined as:
\begin{equation} \label{eq:likelihood}
  L(\theta) =  \lim_{\epsilon \to 0} c_\epsilon Pr(Y_\theta \in B_\epsilon(y_0)) = \lim_{\epsilon \to 0} c_\epsilon \int_{y \in B_\epsilon(y_0)} p(y|\theta)dy
\end{equation}

and the posterior distribution as:
\begin{equation}
p(\theta|y_0) \propto L(\theta)p(\theta)
\end{equation}

Since $p_{y|\theta}$ cannot be evaluated, so does $L(\theta)$ and subsequently $p(\theta|y_0)$.

\subsubsection{Approximate Bayesian Computation (ABC) Rejection Sampling}

ABC Rejection Sampling is a modified version of Rejection Sampling, for cases when likelihood evaluation is intractable. In the Rejection  method a sample is obtained from the prior $\theta \sim p(\theta)$ and it is maintained with probability $L(\theta)/\text{max}_\theta L(\theta)$. The samples $\theta_i$ obtained with this procedure follow the posterior distribution $p(\theta|y_0)$. Although we cannot use this approach out of the box (evaluating $L(\theta)$ is impossible in our case), we can adjust it with some slight modifications.

In the discrete case scenario where $Y_\theta$ can take a finite set of numbers, the likelihood becomes $L(\theta) = Pr(Y_\theta=y_0)$ and the posterior $p(\theta|y_o) \propto Pr(Y_\theta=y_o)p(\theta)$. Hence, we can sample from the prior $\theta_i \sim p(\theta)$, run the simulator $y_i = M(\theta_i, V)$ and maintain $\theta_i$ if only $y_i = y_0$.

The above method becomes less usefull as the finite set of possible $Y_\theta$ values grows large. As the set grows larger, the probability to maintain a sample becomes smaller. In the limit where the set becomes infinite (i.e. continuous case) the probability becomes zero. In order for the method to work in this set-up, a relaxation is introduced; we relax the acceptance criterion by letting $Y_\theta$ lie in a contiguous area around $y_0$, i.e. $Y_\theta \in B_\epsilon(y_0), \epsilon > 0$. The area can be defined as $B_\epsilon(y_0) := \{y: d(y, y_0) < \epsilon \}$ where $d(\cdot, \cdot)$ can represent any valid distance. With this modification, the maintained samples follow an approximate posterior

\begin{equation} \label{eq:approx_posterior}
  p_{d,\epsilon}(\theta|y_0) \propto Pr(Y_\theta \in B_\epsilon(y_0))p(\theta)
  \end{equation}

  where $B_\epsilon$ is defined by $d, \epsilon$. This procedure is called Rejection ABC algorithm and forms the basis of Likelihood-Free methods.

\subsubsection{Summary Statistics}

When the dimensionality of $Y_\theta \in \mathbb{R}^D$ is high, generating samples inside $B_\epsilon(y_0)$ becomes rare; this is the curse of dimensionality. As an representative example if $B_\epsilon(y_0) := \{ y: ||y - y_0||_2^2 < \epsilon^2 \}$ is a hyper-sphere with radius $\epsilon$ and the prior distribution $p(\theta)$ is a uniform distribution in a hyper-cube with side of length $2/epsilon$, the probability of drawing a sample inside the hyper-sphere becomes:

\begin{equation}
  Pr(Y_\theta \in B_\epsilon(y_0)) = Pr(\theta \in B_\epsilon(y_0)) = \frac{V_{hypersphere}}{V_{hypercube}} = \frac{\pi^{D/2}}{D2^{D-1}\Gamma(D/2)} \rightarrow 0 \text{as} D \rightarrow \infty
\end{equation}

We observe that the probality tends to $0$, independently of $\epsilon$; enlarging $\epsilon$ will not increase the acceptance rate. This produces the need for a mapping $T: \mathbb{R}^{D_1} \rightarrow \mathbb{R}^{D_2}$ where $D_1 > D_2$, redefining the area as $B_\epsilon(y_0) := \{y: d(T(y), T(y_0)) < \epsilon \}$. This process is called measuring the distance at the \textit{summary statistics} level.

\subsubsection{Approximation}

Approximating the posterior as $p(\theta|y_0) \approx p_{d,\epsilon}(\theta|y_0) \propto Pr(Y_\theta \in B_\epsilon(y_0))p(\theta)$ where $B_\epsilon(y_0) := \{y: d(T(y), T(y_0)) < \epsilon \}$ introduces two approximation errors:

\begin{itemize}
\item $\epsilon$ is chosen to be large enough for enough samples to be accepted
  \item Summary Statistics (a) make the distance not a metric in a formal sense, i.e. $d = 0$, even if $y \neq y_0$ (b) make possible disjoint sets of $y$ to lie inside $B_\epsilon{y_0}$
  \end{itemize}

  In the following sections we will not use the summary statistics in our expression, for the notation not to clutter. We can state that all the following statements are valid with incorporating summary statistics.
  
  \subsubsection{Optimization Monte Carlo (OMC)}

  We have already defined $B_{d,\epsilon}(y) := \{x: d(y,x)<\epsilon\}$ as the set of points that lie inside area defined by $(d, \epsilon, y)$. Based on that, we can define two useful entities; an indicator function and a conditional distribution.

  \subsubsection*{Indicator Function}

The indicator function $\mathbb{1}_{B_{d,\epsilon}(y)}(x)$ returns 1 if $x \in B_{d,\epsilon}(y)$ and 0 otherwise. If $d(\cdot,\cdot)$ is a formal distance, due to symmetry $\mathbb{1}_{B_{d,\epsilon}(y)}(x) = \mathbb{1}_{B_{d,\epsilon}(x)}(y)$.

\begin{gather} \label{eq:indicator}
  \mathbb{1}_{B_{d,\epsilon}(y)}(x) = \left\{
	\begin{array}{ll}
		1 & \mbox{if } x \in B_{d,\epsilon}(y) \\
		0 & \mbox{else } 
	\end{array} \right. \end{gather}

\subsubsection*{Boxcar Kernel}

The boxcar kernel is the conditional distribution:

\begin{gather}
  p_{d,\epsilon}(y|x) = \left\{
	\begin{array}{ll}
		c  & \mbox{if } d(y,x) \leq \epsilon \\
		0 & \mbox{else } 
	\end{array}
  \right. \text{where} c = \frac{1}{\int_{ \{ y: d(y,x) < \epsilon \} } dy}
\end{gather}
%
If we understand the boxcar kernel as a data generation process we can make two important notices:

\begin{itemize}
\item given a specific $x$, all values $y: y \in B_{d,\epsilon}(x)$ have equal probability to be generated
  \item if a specific $y$ value has been generated, all $x: x \in B_{d,\epsilon}(y)$ have equal probability to be the conditional value that lead to this generation
  \end{itemize}
%
Finally, we can also observe that the kernel can be defined through the indicator function:

\begin{equation}
  p_{d,\epsilon}(y|x) = c \mathbb{1}_{B_{d,\epsilon}(y)}(x) = c \mathbb{1}_{B_{d,\epsilon}(x)}(y)
\end{equation}

\subsubsection*{Initial View}

Based on the knowledge we have so far, we could define and approximate the approximate likelihood $L_{d,\epsilon}(\theta)$ and through~\ref{eq:approx_posterior} the approximate posterior $p_{d, \epsilon}(\theta|y_0)$. The approximation is given below:

\begin{gather} \label{eq:primal_view}
  L_{d, \epsilon}(\theta)=\int_{B_\epsilon(y_0)}p(y|\theta)dy = \int p_{d,\epsilon}(y_0|y)p(y|\theta)dy\\
  \approx \frac{1}{N} \sum_i^N p_{d,\epsilon} (y_0|y_i) \\
  \approx \frac{c}{N} \sum_i^N \mathbb{1}_{B_{d,\epsilon}(y_i)} (y_0), y_i \sim M_r(\theta)
\end{gather}
%
This approach is quite intuitive; approximating likelihood of a specific $\theta$ requires sampling from the data generator and count the fraction of samples that lie inside the area around the observed data. On the other hand, it has a major disadvantage; evaluating $L_{d,\epsilon}(\theta)$ requires resampling $N$ points and checking which ones are close to the observed data $y_0$.

\subsubsection*{Alternative View}

OMC attempts an alternative view to the approximation of $L_{d,\epsilon}(\theta)$, which has some advantages. Instead of incorporating the random generator $M_r(\theta)$, it samples all the nuisance variables from a prior distribution $v_i \sim p(v)$ and then it uses the deterministic mapping $M_d(\theta, v_i)$. More formally the approach is the following:

\begin{gather} 
  L_{d,\epsilon}(\theta)=\int_{B_\epsilon(y_0)}p(y|\theta)dy = \int p_{d,\epsilon}(y_0|y)p(y|\theta)dy\\
  = \int_y \int_v p_{d,\epsilon}(y_0|y)p(y|\theta, v) p(v)dxdv \\
  = \int_v p_{d,\epsilon}(y_0|y=M_d(\theta, v)) p(v)dv \\
  \approx \frac{1}{N} \sum_i^N p_{d,\epsilon} (y_0|y=M_d(\theta, v_i)) \\
  \approx \frac{c}{N} \sum_i^N \mathbb{1}_{B_{d,\epsilon}(M_d(\theta, v_i))} (y_0), v_i \sim p(v)
  \label{eq:alt_view}
\end{gather}
%
Based on this approach, the unnormalized approximate posterior can be defined as:

\begin{equation} \label{eq:posterior}
  p_{d,\epsilon}(\theta|y_0) \propto p(\theta) \sum_i^N \mathbb{1}_{B_{d,\epsilon}(M_d(\theta, v_i))} (y_0)
  \end{equation}
%
Forming an analogy with the previous approach, we sample many nuisance variables in order to absorb the randomness of the generator and we count the fraction of times the deterministic generator produces mapps to outputs close to the observed data. Though it is conceptually close to the previous approach, this approach has a major advantage; we can sample the nuisance variables once (training part) and afterwards evaluate every $\theta$ based on a predefined expression (inference part).


\subsection{Robust Optimisation Monte Carlo (ROMC) approach}
\label{subsec:ROMC}
% The simplifications introduced by OMC, although quite usefull from a
computational point of view, they suffer from some major failure modes:

\begin{itemize}
\item The whole set of acceptable $S_i$ for each nuisance variable
  gets shrinked to a single point$\thetab_i^*$; this can add
  significant error when then the area $S_i$ is relatively big.
\item The weight $w_i$ is computed using only \textbf{local}
  information, i.e.\ the curvature of $g_i$ at the point
  $\theta_i^*$. This approach can introduce significant error when
  $g_i$ is almost flat at $\theta_i^*$, leading to a
  $\text{det}(\jac_i^{*T}\jac_i^*) \rightarrow 0 \Rightarrow w_i
  \rightarrow \infty$, dominating the posterior.
\item There is no way to solve the optimisation problem
  $\thetab_i^* = \text{argmin}_{\thetab} \: [g_i(\thetab)]$ when $g_i$
  is not differentiable.
\end{itemize}


\subsubsection{Sampling and computing expectation in ROMC}

The ROMC approach resolves the aforementioned issues. Instead of
collapsing the acceptance regions into single points, it tries to
approximate them with a bounding box and then define a uniform
distribution on them.\footnote{The description on how to estimate the
  Bounding Box is given in the following chapters.}. This distribution
serves as a proposal distribution for importance sampling. If we
define as $q_i$, the uniform distribution defined on the $i-th$
bounding box, weighted sampling is performed as:

\begin{gather}
  \label{eq:sampling}
  \thetab_{ij} \sim q_i \\
  w_{ij} = \frac{\indicator{\region(\data)}(M_d(\thetab_{ij}, \vb_i)) p(\thetab_{ij})}{q(\thetab_{ij})}
\end{gather}

\noindent
Having defined the procedure for obtaining weighted samples, any expectation $E_{p(\thetab|\data)}[h(\thetab)]$, can be approximated as,

\begin{equation} \label{eq:expectation}
  E_{p(\thetab|\data)}[h(\thetab)] \approx \frac{\sum_{ij} w_{ij} h(\thetab_{ij})}{\sum_{ij} w_{ij}}
\end{equation}


\subsubsection{Construction of the proposal region}

In this section we will describe mathematically the steps needed for
computing the proposal distributions $q_i$. There will be also
presented a Bayesian Optimisation alternative when gradients are not
available.

\subsubsection*{Define and solve deterministic optimisation problems}

For each set of nuisance variables $\vb_i, i = \{1,2,...,n_1 \}$ a
deterministic function is defined as
$f_i(\thetab) = M_d(\thetab,\vb_i)$. For constructing the proposal
region, we search for a point
$\thetab_* : d(f_i(\thetab_*), \data) < \epsilon$; this point can be
obtained by solving the the following optimisation problem:

\begin{subequations}
\begin{alignat}{2}
&\!\min_{\thetab}        &\qquad& g_i(\thetab) = d(\data,  f_i(\thetab))\label{eq:optProb}\\
&\text{subject to} &      & g_i(\thetab) \leq \epsilon
\end{alignat}
\end{subequations}
%
We maintain a list of the solutions $\thetab_i^*$ of the optimisation
problems. If for a specific set of nuisance variables $\vb_i$, there
is no feasible solution we add nothing to the list. The optimisation
problem can be treated as unconstrained, accepting the optimal point
$\thetab_i^* = \text{argmin}_{\thetab} g_i(\thetab)$ only if
$g_i(\thetab_i^*) < \epsilon$.

\subsubsection*{Gradient-Based Approach}
\label{subsubsec:GB_approach}

The nature of the generative model $M_r(\thetab)$, specifies the
properties of the objective function $g_i$. If $g_i$ is continuous
with smooth gradients $\nabla_{\thetab} g_i$ any gradient-based
iterative algorithm can be used for solving~\ref{eq:optProb}. The
gradients $\nabla_{\thetab} g_i$ can be either provided in closed form
or approximated by finite differences.

\subsubsection*{Bayesian Optimisation Approach}
\label{subsubsec:GP_approach}

In cases where the gradients are not available, the Bayesian
Optimisation scheme provides an alternative
choice~\autocite{Shahriari2016}. With this approach, apart from
obtaining an optimal $\thetab_i^* $, a surrogate model $\hat{d}_i$ of
the distance $g_i$ is fitted; this approximate model can be used in
the following steps for making the method more
efficient. Specifically, in the construction of the proposal region
and in
equations~\eqref{eq:approx_posterior},~\eqref{eq:sampling},~\eqref{eq:expectation}
it could replace $g_i$ in the evaluation of the indicator
function~\ref{eq:indicator}, providing a major speed-up.

\subsubsection*{Construction of the proposal area $q_i$}

After obtaining $\thetab_i^*$ such that $g_i(\thetab_i^*) < \epsilon$,
we need to construct a bounding box around it. The bounding box must
contain the acceptance region around $\thetab_i^*$, i.e.\
$\{ \thetab : g_i(\thetab) < \epsilon$ and
$d(\thetab, \thetab_i^*) < M \}$. The second condition
$d(\thetab, \thetab_i^*) < M$ is meant to describe that if
$S_i := \{ \thetab : g_i(\thetab) < \epsilon \} $ contains a number of
disjoint sets of $\thetab$ that respect $g_i(\thetab) < \epsilon$, we
want our bounding box to fit the one that contains $\thetab_i^*$. We
seek for a bounding box to be as tight as possible to the local
acceptance region (enlarging the bounding box dicreases the acceptance
rate) but large enough for not discarding accepted areas.

In contrast with the OMC approach, we construct the bounding box by
querying the indication function along the search directions. The
search directions $\mathbf{v}_d$ are computed as the eigenvectors of
the curvature at $\thetab_i^*$ and a line-search method is used to
obtain the limit point where
$g_i(\thetab_i^* + \kappa \vb_d) \geq
\epsilon$. Algorithm~\ref{alg:region_construction} describes
analytically the method. After the limits are obtained along all
search directions, we define the uniform distribution $q_i$ on the
bounding box. This is the proposal distribution of the importance
sampling explained in \eqref{eq:sampling}.

The simplifications introduced by OMC, although quite usefull from a
computational point of view, they suffer from some major failure modes:

\begin{itemize}
\item The whole set of acceptable $S_i$ for each nuisance variable
  gets shrinked to a single point$\thetab_i^*$; this can add
  significant error when then the area $S_i$ is relatively big.
\item The weight $w_i$ is computed using only \textbf{local}
  information, i.e.\ the curvature of $g_i$ at the point
  $\theta_i^*$. This approach can introduce significant error when
  $g_i$ is almost flat at $\theta_i^*$, leading to a
  $\text{det}(\jac_i^{*T}\jac_i^*) \rightarrow 0 \Rightarrow w_i
  \rightarrow \infty$, dominating the posterior.
\item There is no way to solve the optimisation problem
  $\thetab_i^* = \text{argmin}_{\thetab} \: [g_i(\thetab)]$ when $g_i$
  is not differentiable.
\end{itemize}


\subsubsection{Sampling and computing expectation in ROMC}

The ROMC approach resolves the aforementioned issues. Instead of
collapsing the acceptance regions into single points, it tries to
approximate them with a bounding box and then define a uniform
distribution on them.\footnote{The description on how to estimate the
  Bounding Box is given in the following chapters.}. This distribution
serves as a proposal distribution for importance sampling. If we
define as $q_i$, the uniform distribution defined on the $i-th$
bounding box, weighted sampling is performed as:

\begin{gather}
  \label{eq:sampling}
  \thetab_{ij} \sim q_i \\
  w_{ij} = \frac{\indicator{\region(\data)}(M_d(\thetab_{ij}, \vb_i)) p(\thetab_{ij})}{q(\thetab_{ij})}
\end{gather}

\noindent
Having defined the procedure for obtaining weighted samples, any expectation $E_{p(\thetab|\data)}[h(\thetab)]$, can be approximated as,

\begin{equation} \label{eq:expectation}
  E_{p(\thetab|\data)}[h(\thetab)] \approx \frac{\sum_{ij} w_{ij} h(\thetab_{ij})}{\sum_{ij} w_{ij}}
\end{equation}


\subsubsection{Construction of the proposal region}

In this section we will describe mathematically the steps needed for
computing the proposal distributions $q_i$. There will be also
presented a Bayesian Optimisation alternative when gradients are not
available.

\subsubsection*{Define and solve deterministic optimisation problems}

For each set of nuisance variables $\vb_i, i = \{1,2,...,n_1 \}$ a
deterministic function is defined as
$f_i(\thetab) = M_d(\thetab,\vb_i)$. For constructing the proposal
region, we search for a point
$\thetab_* : d(f_i(\thetab_*), \data) < \epsilon$; this point can be
obtained by solving the the following optimisation problem:

\begin{subequations}
\begin{alignat}{2}
&\!\min_{\thetab}        &\qquad& g_i(\thetab) = d(\data,  f_i(\thetab))\label{eq:optProb}\\
&\text{subject to} &      & g_i(\thetab) \leq \epsilon
\end{alignat}
\end{subequations}
%
We maintain a list of the solutions $\thetab_i^*$ of the optimisation
problems. If for a specific set of nuisance variables $\vb_i$, there
is no feasible solution we add nothing to the list. The optimisation
problem can be treated as unconstrained, accepting the optimal point
$\thetab_i^* = \text{argmin}_{\thetab} g_i(\thetab)$ only if
$g_i(\thetab_i^*) < \epsilon$.

\subsubsection*{Gradient-Based Approach}
\label{subsubsec:GB_approach}

The nature of the generative model $M_r(\thetab)$, specifies the
properties of the objective function $g_i$. If $g_i$ is continuous
with smooth gradients $\nabla_{\thetab} g_i$ any gradient-based
iterative algorithm can be used for solving~\ref{eq:optProb}. The
gradients $\nabla_{\thetab} g_i$ can be either provided in closed form
or approximated by finite differences.

\subsubsection*{Bayesian Optimisation Approach}
\label{subsubsec:GP_approach}

In cases where the gradients are not available, the Bayesian
Optimisation scheme provides an alternative
choice~\autocite{Shahriari2016}. With this approach, apart from
obtaining an optimal $\thetab_i^* $, a surrogate model $\hat{d}_i$ of
the distance $g_i$ is fitted; this approximate model can be used in
the following steps for making the method more
efficient. Specifically, in the construction of the proposal region
and in
equations~\eqref{eq:approx_posterior},~\eqref{eq:sampling},~\eqref{eq:expectation}
it could replace $g_i$ in the evaluation of the indicator
function~\ref{eq:indicator}, providing a major speed-up.

\subsubsection*{Construction of the proposal area $q_i$}

After obtaining $\thetab_i^*$ such that $g_i(\thetab_i^*) < \epsilon$,
we need to construct a bounding box around it. The bounding box must
contain the acceptance region around $\thetab_i^*$, i.e.\
$\{ \thetab : g_i(\thetab) < \epsilon$ and
$d(\thetab, \thetab_i^*) < M \}$. The second condition
$d(\thetab, \thetab_i^*) < M$ is meant to describe that if
$S_i := \{ \thetab : g_i(\thetab) < \epsilon \} $ contains a number of
disjoint sets of $\thetab$ that respect $g_i(\thetab) < \epsilon$, we
want our bounding box to fit the one that contains $\thetab_i^*$. We
seek for a bounding box to be as tight as possible to the local
acceptance region (enlarging the bounding box dicreases the acceptance
rate) but large enough for not discarding accepted areas.

In contrast with the OMC approach, we construct the bounding box by
querying the indication function along the search directions. The
search directions $\mathbf{v}_d$ are computed as the eigenvectors of
the curvature at $\thetab_i^*$ and a line-search method is used to
obtain the limit point where
$g_i(\thetab_i^* + \kappa \vb_d) \geq
\epsilon$. Algorithm~\ref{alg:region_construction} describes
analytically the method. After the limits are obtained along all
search directions, we define the uniform distribution $q_i$ on the
bounding box. This is the proposal distribution of the importance
sampling explained in \eqref{eq:sampling}.


\subsection{Algorithmic description of ROMC}
\label{subsec:romc-algorithmic}
% In this section, we attempt the depiction of the mathematical
description of ROMC in algorithms. Specificaly, in chapter
\ref{subsubsec:romc-meta-algorithm} we present the general algorithmic
description of ROMC as a meta-algorithm and in chapter
\ref{subsubsec:alg-training-inference} the proposals of ROMC for
solving the training and the inference parts.

\subsubsection{ROMC as a Meta-Algorithm}
\label{subsubsec:romc-meta-algorithm}

As stated at the introductory chapter, ROMC can be understood as
step-by-step alorithmic approach for perfroming the inference in
Simulator-Based Models. The particular methods used for solving the
sub-tasks are left as a free choice to the user. As presented in
Algorithm~\ref{alg:meta_alg}, the methods involved in solving the
optimistation problem (step~\ref{algstep:optimise}) and constructing
the bounding box (step~\ref{algstep:bounding_box}) are not
restricted. The practiosioner may choose any convenient algorithm,
judging the trade-offs between accuracy, robustness, efficiency and
complexity. In particular for the optimisation step, the choice of the
appropriate optimiser should also consider the properties of
$g_i(\thetab)$. Some important questions that should be considered are
whether the function differentiable and if so whether we know the
gradients $\nabla_{\thetab} [g_i] $ in closed-form. As described in
sections~\ref{subsubsec:GB_approach} and~\ref{subsubsec:GP_approach},
ROMC proposes two alternative optimisation schemes (gradient-based and
gaussian-process approach) depending on whether the gradients are
available or not.

\begin{algorithm}[t]
	\caption{ROMC as a Meta-Algorithm. Requires $M_r(\theta), y_0$. Hyperparameters $n_1,n_2$.}\label{alg:meta_alg}
	\begin{algorithmic}[1]
		\For{$i \gets 1 \textrm{ to } n_1$}
    \State Sample a random state $\vb_i \sim p(\vb)$
		\State Define the deterministic mapping $f_i(\thetab) = M_d(\thetab, \vb)$ and therefore $g_i(\thetab) = d(f_i(\theta), y_0)$.
    \State Obtain $d_i^* = \text{min}_{\thetab} \: [g_i(\thetab)]$ and $\thetab_i^* = \text{argmin}_{\thetab}\: [g_i(\thetab)]$ using any convenient optimiser. \label{algstep:optimise}
    \State Approximate the local area $\{ \thetab : g_i(\thetab) < \epsilon$ and $d(\thetab, \thetab_i^*) < M \}$ with a Bounding Box, using any convenient method. \label{algstep:bounding_box}
		\State Define a uniform distribution $q_i(\thetab)$ over the Bounding Box.
			\For{$j \gets 1 \textrm{ to } n_2$}
			\State $\thetabij \sim q_i(\thetab)$
			\State Accept $\thetabij$ as posterior sample with weight $w_{ij} = \frac{p(\thetabij)}{q_i(\thetabij)} \indicator{\regioni} (\thetabij)$
			\EndFor
      \EndFor
     \Return(List with samples $\thetabij$ and weights $w_{ij}$) 
	\end{algorithmic}
\end{algorithm}


\subsubsection{Training and Inference Algorithms}
\label{subsubsec:alg-training-inference}

In this section, we will provide the algorithmic description of the
ROMC method; (a) the procedures for solving the optimisation problems
using either the gradient based approach or the Gaussian Process
alternative and (b) the construction of the Bounding Box. Afterwards,
we will discuss the advantages and the disadvantages of each choice
both in terms of accuracy and efficiency.

\noindent
At a high-level, the ROMC method can be split into the training and
the inference part.

\noindent
At the training (fitting) part, the goal is the estimation of the
proposal regions $q_i$. The steps include (a) sampling the nuisance
variables $\vb_i \sim p(\vb)$ (b) defining the optimisation problems
$\min_{\thetab} [g_i(\thetab)]$ (c) obtaining $\thetab_i^*$ (d) checking
whether $d_i^* < \epsilon$ and (e) building the bounding box for
obtaining the proposal region $q_i$. If gradients are available, using
a gradient-based method is adviced for obtaining $\thetab_i^*$ much
faster.  Providing $\nabla_{\thetab} g_i$ in closed-form provides an
upgrade in both accuracy and efficiency; If closed-form description is
not available, approximate gradients with finite-differences
$\frac{\partial g_i(\thetab)}{\thetab_d} = \frac{g_i(\thetab_d + h
  \mathbf{e_d}) - g_i(\thetab_d)}{h}$ requires two evaluations of $g_i$ for
\textbf{every} parameter $\thetab_d$. For low-dimensional problems
though, this approach still works well. When gradients are not
available or $g_i$ is not differentiable, using the Gaussian Process
is the only solution. In this case, the training part is much slower
due to the fitting of the surrogate model and the ignorance of the
slope throughout the optimisation procedure. Nevertheless, computing
the proposal region $q_i$ becomes faster since $\hat{d}_i$ can be used
instead of $g_i$ which involves running the whole simulator
$M_d(\thetab, \vb_i)$ for each query. The algorithms are presented
in~\ref{alg:training_GB} and~\ref{alg:training_GP}.

\noindent
Performing the inference includes (a) evaluating the unnormalised
posterior $p_{d, \epsilon}(\theta_b|\data)$ (b) sampling from the
posterior $ \thetab_i \sim p_{d, \epsilon}(\theta_b|\data)$ and (c)
computing an expectation $E_{\thetab|\data}[h(\thetab)]$.  Computing
an expectation can be done easily after weighted samples are obtained
\ref{eq:expectation}, so we will not discuss it seperately.

\noindent
For evaluating the unnormalized posterior in the gradient-based
approach, only the deterministic functions $g_i$ and the prior
distribution $p(\thetab)$ are required; there is no need for solving
the optimisation problems and building the proposal regions. The
evaluation requires iterating over all $g_i$ and evaluating the
distance from the observed data. In contrast, using the GP approach,
the optimisation part should be performed first for fitting the
surrogate models $\hat{d}_i(\thetab)$ and evaluate the indicator
function on them. This provides an important speed-up, especially when
running the simulator is computationally expensive. The evaluation of
the posterior is presented analytically in~\ref{alg:posterior_GB}
and~\ref{alg:posterior_GP}.

\noindent
Sampling is performed by getting $n_2$ samples from each proposal
region $q_i$. For each sample $\thetab_{ij}$, the indicator function
is evaluated $\indicator{\regioni(\data)}(\thetab_{ij})$ for checking
if it lies inside the acceptance region. If so the corresponding
weight is computed as in \autocite{eq:sampling}. As before, if a surrogate
model $\hat{d}$ has been fitted, it can be used for the evaluation of
the indicator function providing again a speedup. Apparently, the
compuational benefit is more important compared to posterior
evaluation, because the indicator function must be evaluated for a
total of $n_1 \times n_2$ points. The sampling algorithms are
presented step-by-step in~\ref{alg:sampling_GB}
and~\ref{alg:sampling_GP}.

\noindent
As a conclusion, we can state that the choise of using a bayesian
optimisation approach provides a significant speed-up in the inference
part with the cost of making the training part slower and a possible
approximation error. It is typical in many Machine-Learning use cases,
being able to provide enough time and computational resources for the
training phase, but asking for efficiency in the inference
part. Having that in mind, we can say that the Gaussian-Process is a
quite usefull alternative.

\begin{minipage}{0.46\textwidth}
\begin{algorithm}[H]
    \centering
    \caption{Training Part - Gradient approach. Requires $g_i(\theta), p(\theta)$}\label{alg:training_GB}
    \begin{algorithmic}[1]
      \For{$i \gets 1 \textrm{ to } n$}
        \State Obtain $\theta_i^*$ using a Gradient Optimiser
        \If{$g_i(\theta_i^*) > \epsilon$}
        \State{go to} 1
        \Else
        \State Approximate $H_i \approx J^T_iJ_i$
        \State Use algorihm~\ref{alg:region_construction} to obtain $q_i$
        \EndIf      
      \EndFor
      \Return{$q_i, p(\theta), g_i(\theta)$}
    \end{algorithmic}
\end{algorithm}
\end{minipage}
\hfill
\begin{minipage}{0.46\textwidth}
\begin{algorithm}[H]
    \centering
    \caption{Training Part - GP approach. Requires $g_i(\theta), p(\theta)$}\label{alg:training_GP}
    \begin{algorithmic}[1]
      \For{$i \gets 1 \textrm{ to } n$}
        \State Obtain $\theta_i^*, \hat{d}_i(\theta)$ using a GP approach
        \If{$g_i(\theta_i^*) > \epsilon$}
        \State{go to} 1
        \Else
        \State Approximate $H_i \approx J^T_iJ_i$
        \State Use algorihm~\ref{alg:region_construction} to obtain $q_i$
        \EndIf      
      \EndFor
      \Return{$q_i, p(\theta), \hat{d}_i(\theta)$}
    \end{algorithmic}
\end{algorithm}
\end{minipage}

\begin{algorithm}[!ht]
	\caption{Proposal Region $q_i$ construction; Needs, a model of distance $d$ ($\hat{d}$ or $g_i$), optimal point $\theta_i^*$, number of refinements $K$, step size $\eta$ and curvature matrix $\hessian_i$ ($J_i^TJ_i $ or GP Hessian)}\label{alg:region_construction}
	\begin{algorithmic}[1]
	\State Compute eigenvectors $\mathbf{v}_{d}$ of $H_i$ {\scriptsize ($d = 1,\ldots,||\theta ||)$}
	\For{$d \gets 1 \textrm{ to } ||\theta||$}
		\State $\Tilde{\theta} \gets \theta_i^*$ \label{algstep:box_constr_start}
		\State $k \gets 0$
		\Repeat
        	\Repeat
                \State $\Tilde{\theta} \gets \Tilde{\theta} + \eta \ \mathbf{v}_{d}$ \Comment{Large step size $\eta$.}
        	\Until{$d( (\Tilde{\theta}, i), ) \ge \epsilon$}
        	\State $\Tilde{\theta} \gets \Tilde{\theta} - \eta \ \mathbf{v}_{d}$
        	\State $\eta \gets \eta/2$ \Comment{More accurate region boundary}
        	\State $k \gets k + 1$
    	\Until $k = K$
    	\State Set final $\Tilde{\theta}$ as region end point. \label{algstep:box_constr_end}
    	\State Repeat steps~\ref{algstep:box_constr_start}~-~\ref{algstep:box_constr_end} for $\mathbf{v}_{d} = - \mathbf{v}_{d}$
	\EndFor
	\State Fit a rectangular box around the region end points and define $q_i$ as uniform distribution
	\end{algorithmic}
\end{algorithm}

\begin{minipage}{0.46\textwidth}
\begin{algorithm}[H]
    \centering
    \caption{Evaluate unnormalised posterior - Gradient approach. Requires $g_i(\theta), p(\theta)$}\label{alg:posterior_GB}
    \begin{algorithmic}[1]
      \State $k \leftarrow 0$
        \For {$i \gets 1 \textrm{ to } n_1$}
          \If {$g_i(\theta) > \epsilon$}
            \State $k \leftarrow k + 1$
          \EndIf
          \EndFor
      \Return{$kp(\theta)$}
    \end{algorithmic}
\end{algorithm}
\end{minipage}
\hfill
\begin{minipage}{0.46\textwidth}
\begin{algorithm}[H]
    \centering
    \caption{Evaluate unnormalised posterior - GP approach. Requires $\hat{d}_i(\theta), p(\theta)$}\label{alg:posterior_GP}
    \begin{algorithmic}[1]
      \State $k \leftarrow 0$
        \For {$i \gets 1 \textrm{ to } n_1$}
          \If {$d_i(\theta) > \epsilon$}
            \State $k \leftarrow k + 1$
          \EndIf
          \EndFor
      \Return{$kp(\theta)$}
    \end{algorithmic}
\end{algorithm}
\end{minipage}


\begin{minipage}{0.46\textwidth}
\begin{algorithm}[H]
    \centering
    \caption{Sampling - Gradient Based approach. Requires $g_i(\theta), p(\theta), q_i$}\label{alg:sampling_GB}
    \begin{algorithmic}[1]
      \For {$i \gets 1 \textrm{ to } n_1$}
      \For {$j \gets 1 \textrm{ to } n_2$}
          \State $\theta_{ij} \sim q_i$
          \If {$g_i(\theta_{ij}) > \epsilon$}
            \State Reject $\theta_{ij}$
          \Else {}
            \State $w_{ij} = \frac{p(\theta_{ij})}{q(\theta_{ij})}$
            \State Accept $\theta_{ij}$, with weight $w_{ij}$
          \EndIf
      \EndFor
      \EndFor
    \end{algorithmic}
\end{algorithm}
\end{minipage}
\hfill
\begin{minipage}{0.46\textwidth}
\begin{algorithm}[H]
    \centering
    \caption{Sampling - GP approach. Requires $\hat{d}_i(\theta), p(\theta), q_i$}\label{alg:sampling_GP}
    \begin{algorithmic}[1]
      \For {$i \gets 1 \textrm{ to } n_1$}
      \For {$j \gets 1 \textrm{ to } n_2$}
          \State $\theta_{ij} \sim q_i$
          \If {$\hat{d}_i(\theta_{ij}) > \epsilon$}
            \State Reject $\theta_{ij}$
          \Else {}
            \State $w_{ij} = \frac{p(\theta_{ij})}{q(\theta_{ij})}$
            \State Accept $\theta_{ij}$, with weight $w_{ij}$
          \EndIf
      \EndFor
      \EndFor
    \end{algorithmic}
\end{algorithm}
\end{minipage}

In this section, we attempt the depiction of the mathematical
description of ROMC in algorithms. Specificaly, in chapter
\ref{subsubsec:romc-meta-algorithm} we present the general algorithmic
description of ROMC as a meta-algorithm and in chapter
\ref{subsubsec:alg-training-inference} the proposals of ROMC for
solving the training and the inference parts.

\subsubsection{ROMC as a Meta-Algorithm}
\label{subsubsec:romc-meta-algorithm}

As stated at the introductory chapter, ROMC can be understood as
step-by-step alorithmic approach for perfroming the inference in
Simulator-Based Models. The particular methods used for solving the
sub-tasks are left as a free choice to the user. As presented in
Algorithm~\ref{alg:meta_alg}, the methods involved in solving the
optimistation problem (step~\ref{algstep:optimise}) and constructing
the bounding box (step~\ref{algstep:bounding_box}) are not
restricted. The practiosioner may choose any convenient algorithm,
judging the trade-offs between accuracy, robustness, efficiency and
complexity. In particular for the optimisation step, the choice of the
appropriate optimiser should also consider the properties of
$g_i(\thetab)$. Some important questions that should be considered are
whether the function differentiable and if so whether we know the
gradients $\nabla_{\thetab} [g_i] $ in closed-form. As described in
sections~\ref{subsubsec:GB_approach} and~\ref{subsubsec:GP_approach},
ROMC proposes two alternative optimisation schemes (gradient-based and
gaussian-process approach) depending on whether the gradients are
available or not.

\begin{algorithm}[t]
	\caption{ROMC as a Meta-Algorithm. Requires $M_r(\theta), y_0$. Hyperparameters $n_1,n_2$.}\label{alg:meta_alg}
	\begin{algorithmic}[1]
		\For{$i \gets 1 \textrm{ to } n_1$}
    \State Sample a random state $\vb_i \sim p(\vb)$
		\State Define the deterministic mapping $f_i(\thetab) = M_d(\thetab, \vb)$ and therefore $g_i(\thetab) = d(f_i(\theta), y_0)$.
    \State Obtain $d_i^* = \text{min}_{\thetab} \: [g_i(\thetab)]$ and $\thetab_i^* = \text{argmin}_{\thetab}\: [g_i(\thetab)]$ using any convenient optimiser. \label{algstep:optimise}
    \State Approximate the local area $\{ \thetab : g_i(\thetab) < \epsilon$ and $d(\thetab, \thetab_i^*) < M \}$ with a Bounding Box, using any convenient method. \label{algstep:bounding_box}
		\State Define a uniform distribution $q_i(\thetab)$ over the Bounding Box.
			\For{$j \gets 1 \textrm{ to } n_2$}
			\State $\thetabij \sim q_i(\thetab)$
			\State Accept $\thetabij$ as posterior sample with weight $w_{ij} = \frac{p(\thetabij)}{q_i(\thetabij)} \indicator{\regioni} (\thetabij)$
			\EndFor
      \EndFor
     \Return(List with samples $\thetabij$ and weights $w_{ij}$) 
	\end{algorithmic}
\end{algorithm}


\subsubsection{Training and Inference Algorithms}
\label{subsubsec:alg-training-inference}

In this section, we will provide the algorithmic description of the
ROMC method; (a) the procedures for solving the optimisation problems
using either the gradient based approach or the Gaussian Process
alternative and (b) the construction of the Bounding Box. Afterwards,
we will discuss the advantages and the disadvantages of each choice
both in terms of accuracy and efficiency.

\noindent
At a high-level, the ROMC method can be split into the training and
the inference part.

\noindent
At the training (fitting) part, the goal is the estimation of the
proposal regions $q_i$. The steps include (a) sampling the nuisance
variables $\vb_i \sim p(\vb)$ (b) defining the optimisation problems
$\min_{\thetab} [g_i(\thetab)]$ (c) obtaining $\thetab_i^*$ (d) checking
whether $d_i^* < \epsilon$ and (e) building the bounding box for
obtaining the proposal region $q_i$. If gradients are available, using
a gradient-based method is adviced for obtaining $\thetab_i^*$ much
faster.  Providing $\nabla_{\thetab} g_i$ in closed-form provides an
upgrade in both accuracy and efficiency; If closed-form description is
not available, approximate gradients with finite-differences
$\frac{\partial g_i(\thetab)}{\thetab_d} = \frac{g_i(\thetab_d + h
  \mathbf{e_d}) - g_i(\thetab_d)}{h}$ requires two evaluations of $g_i$ for
\textbf{every} parameter $\thetab_d$. For low-dimensional problems
though, this approach still works well. When gradients are not
available or $g_i$ is not differentiable, using the Gaussian Process
is the only solution. In this case, the training part is much slower
due to the fitting of the surrogate model and the ignorance of the
slope throughout the optimisation procedure. Nevertheless, computing
the proposal region $q_i$ becomes faster since $\hat{d}_i$ can be used
instead of $g_i$ which involves running the whole simulator
$M_d(\thetab, \vb_i)$ for each query. The algorithms are presented
in~\ref{alg:training_GB} and~\ref{alg:training_GP}.

\noindent
Performing the inference includes (a) evaluating the unnormalised
posterior $p_{d, \epsilon}(\theta_b|\data)$ (b) sampling from the
posterior $ \thetab_i \sim p_{d, \epsilon}(\theta_b|\data)$ and (c)
computing an expectation $E_{\thetab|\data}[h(\thetab)]$.  Computing
an expectation can be done easily after weighted samples are obtained
\ref{eq:expectation}, so we will not discuss it seperately.

\noindent
For evaluating the unnormalized posterior in the gradient-based
approach, only the deterministic functions $g_i$ and the prior
distribution $p(\thetab)$ are required; there is no need for solving
the optimisation problems and building the proposal regions. The
evaluation requires iterating over all $g_i$ and evaluating the
distance from the observed data. In contrast, using the GP approach,
the optimisation part should be performed first for fitting the
surrogate models $\hat{d}_i(\thetab)$ and evaluate the indicator
function on them. This provides an important speed-up, especially when
running the simulator is computationally expensive. The evaluation of
the posterior is presented analytically in~\ref{alg:posterior_GB}
and~\ref{alg:posterior_GP}.

\noindent
Sampling is performed by getting $n_2$ samples from each proposal
region $q_i$. For each sample $\thetab_{ij}$, the indicator function
is evaluated $\indicator{\regioni(\data)}(\thetab_{ij})$ for checking
if it lies inside the acceptance region. If so the corresponding
weight is computed as in \autocite{eq:sampling}. As before, if a surrogate
model $\hat{d}$ has been fitted, it can be used for the evaluation of
the indicator function providing again a speedup. Apparently, the
compuational benefit is more important compared to posterior
evaluation, because the indicator function must be evaluated for a
total of $n_1 \times n_2$ points. The sampling algorithms are
presented step-by-step in~\ref{alg:sampling_GB}
and~\ref{alg:sampling_GP}.

\noindent
As a conclusion, we can state that the choise of using a bayesian
optimisation approach provides a significant speed-up in the inference
part with the cost of making the training part slower and a possible
approximation error. It is typical in many Machine-Learning use cases,
being able to provide enough time and computational resources for the
training phase, but asking for efficiency in the inference
part. Having that in mind, we can say that the Gaussian-Process is a
quite usefull alternative.

\begin{minipage}{0.46\textwidth}
\begin{algorithm}[H]
    \centering
    \caption{Training Part - Gradient approach. Requires $g_i(\theta), p(\theta)$}\label{alg:training_GB}
    \begin{algorithmic}[1]
      \For{$i \gets 1 \textrm{ to } n$}
        \State Obtain $\theta_i^*$ using a Gradient Optimiser
        \If{$g_i(\theta_i^*) > \epsilon$}
        \State{go to} 1
        \Else
        \State Approximate $H_i \approx J^T_iJ_i$
        \State Use algorihm~\ref{alg:region_construction} to obtain $q_i$
        \EndIf      
      \EndFor
      \Return{$q_i, p(\theta), g_i(\theta)$}
    \end{algorithmic}
\end{algorithm}
\end{minipage}
\hfill
\begin{minipage}{0.46\textwidth}
\begin{algorithm}[H]
    \centering
    \caption{Training Part - GP approach. Requires $g_i(\theta), p(\theta)$}\label{alg:training_GP}
    \begin{algorithmic}[1]
      \For{$i \gets 1 \textrm{ to } n$}
        \State Obtain $\theta_i^*, \hat{d}_i(\theta)$ using a GP approach
        \If{$g_i(\theta_i^*) > \epsilon$}
        \State{go to} 1
        \Else
        \State Approximate $H_i \approx J^T_iJ_i$
        \State Use algorihm~\ref{alg:region_construction} to obtain $q_i$
        \EndIf      
      \EndFor
      \Return{$q_i, p(\theta), \hat{d}_i(\theta)$}
    \end{algorithmic}
\end{algorithm}
\end{minipage}

\begin{algorithm}[!ht]
	\caption{Proposal Region $q_i$ construction; Needs, a model of distance $d$ ($\hat{d}$ or $g_i$), optimal point $\theta_i^*$, number of refinements $K$, step size $\eta$ and curvature matrix $\hessian_i$ ($J_i^TJ_i $ or GP Hessian)}\label{alg:region_construction}
	\begin{algorithmic}[1]
	\State Compute eigenvectors $\mathbf{v}_{d}$ of $H_i$ {\scriptsize ($d = 1,\ldots,||\theta ||)$}
	\For{$d \gets 1 \textrm{ to } ||\theta||$}
		\State $\Tilde{\theta} \gets \theta_i^*$ \label{algstep:box_constr_start}
		\State $k \gets 0$
		\Repeat
        	\Repeat
                \State $\Tilde{\theta} \gets \Tilde{\theta} + \eta \ \mathbf{v}_{d}$ \Comment{Large step size $\eta$.}
        	\Until{$d( (\Tilde{\theta}, i), ) \ge \epsilon$}
        	\State $\Tilde{\theta} \gets \Tilde{\theta} - \eta \ \mathbf{v}_{d}$
        	\State $\eta \gets \eta/2$ \Comment{More accurate region boundary}
        	\State $k \gets k + 1$
    	\Until $k = K$
    	\State Set final $\Tilde{\theta}$ as region end point. \label{algstep:box_constr_end}
    	\State Repeat steps~\ref{algstep:box_constr_start}~-~\ref{algstep:box_constr_end} for $\mathbf{v}_{d} = - \mathbf{v}_{d}$
	\EndFor
	\State Fit a rectangular box around the region end points and define $q_i$ as uniform distribution
	\end{algorithmic}
\end{algorithm}

\begin{minipage}{0.46\textwidth}
\begin{algorithm}[H]
    \centering
    \caption{Evaluate unnormalised posterior - Gradient approach. Requires $g_i(\theta), p(\theta)$}\label{alg:posterior_GB}
    \begin{algorithmic}[1]
      \State $k \leftarrow 0$
        \For {$i \gets 1 \textrm{ to } n_1$}
          \If {$g_i(\theta) > \epsilon$}
            \State $k \leftarrow k + 1$
          \EndIf
          \EndFor
      \Return{$kp(\theta)$}
    \end{algorithmic}
\end{algorithm}
\end{minipage}
\hfill
\begin{minipage}{0.46\textwidth}
\begin{algorithm}[H]
    \centering
    \caption{Evaluate unnormalised posterior - GP approach. Requires $\hat{d}_i(\theta), p(\theta)$}\label{alg:posterior_GP}
    \begin{algorithmic}[1]
      \State $k \leftarrow 0$
        \For {$i \gets 1 \textrm{ to } n_1$}
          \If {$d_i(\theta) > \epsilon$}
            \State $k \leftarrow k + 1$
          \EndIf
          \EndFor
      \Return{$kp(\theta)$}
    \end{algorithmic}
\end{algorithm}
\end{minipage}


\begin{minipage}{0.46\textwidth}
\begin{algorithm}[H]
    \centering
    \caption{Sampling - Gradient Based approach. Requires $g_i(\theta), p(\theta), q_i$}\label{alg:sampling_GB}
    \begin{algorithmic}[1]
      \For {$i \gets 1 \textrm{ to } n_1$}
      \For {$j \gets 1 \textrm{ to } n_2$}
          \State $\theta_{ij} \sim q_i$
          \If {$g_i(\theta_{ij}) > \epsilon$}
            \State Reject $\theta_{ij}$
          \Else {}
            \State $w_{ij} = \frac{p(\theta_{ij})}{q(\theta_{ij})}$
            \State Accept $\theta_{ij}$, with weight $w_{ij}$
          \EndIf
      \EndFor
      \EndFor
    \end{algorithmic}
\end{algorithm}
\end{minipage}
\hfill
\begin{minipage}{0.46\textwidth}
\begin{algorithm}[H]
    \centering
    \caption{Sampling - GP approach. Requires $\hat{d}_i(\theta), p(\theta), q_i$}\label{alg:sampling_GP}
    \begin{algorithmic}[1]
      \For {$i \gets 1 \textrm{ to } n_1$}
      \For {$j \gets 1 \textrm{ to } n_2$}
          \State $\theta_{ij} \sim q_i$
          \If {$\hat{d}_i(\theta_{ij}) > \epsilon$}
            \State Reject $\theta_{ij}$
          \Else {}
            \State $w_{ij} = \frac{p(\theta_{ij})}{q(\theta_{ij})}$
            \State Accept $\theta_{ij}$, with weight $w_{ij}$
          \EndIf
      \EndFor
      \EndFor
    \end{algorithmic}
\end{algorithm}
\end{minipage}


\subsection{Engine for Likelihood-Free Inference (ELFI) package}
\label{subsec:elfi}
% The Engine for Likelihood-Free Inference (ELFI) \cite{1708.00707} is a
Python software library dedicated to likelihood-free inference
(LFI). ELFI models in a convenient manner all the fundamental
components of a Probabilistic Model such as priors, simulators,
summaries and distances. Furthermore, ELFI already supports a range of
likelihood-free inference methods proposed in the last years.

\subsubsection{Modelling}
\label{sec:modelling}

ELFI models the Probabilistic Model as a Directed Acyclic Graph (DAG);
it implements this functionality based on the package NetworkX, which
is designed for creating general purpose graphs. Although not
restricted to that, in most cases the structure of a likelihood-free
model follows the pattern presented in figure~\ref{fig:elfi-model};
there are edges that connect the \textit{prior} distributions to the
simulator, the simulator is connected to the summary statistics which
is turn are connected to the distance. The distance is the output
node. Samples can be obtained from all nodes through sequential
sampling. The nodes that are defined as
\textit{elfi.Prior}\footnote{The \textit{elfi.Prior} functionality is
  a wrapper around the scipy.stats package.} are automatically
considered as the parameters of interest and they are the only nodes
that, apart from sampling, they should also provide pdf
evaluation. The function passed as argument in the
\textit{elfi.Summary} node can be any valid Python function with
arguments the prior variables. Finally, the observations should be
passed in the appropriate node through the argument
\pinline{observed}; all the nodes afterwards are evaluated at the
observations as well.

\begin{figure}[!ht]
    \begin{center}
      \includegraphics[width=0.8\textwidth]{./Thesis/images/chapter2/elfi.png}
    \end{center}
    \caption{Image taken from \cite{1708.00707}}
    \label{fig:elfi-model}
\end{figure}


\subsubsection{Inference Methods}
\label{sec:inference-methods}

The inference Methods implemented at the ELFI follow some common
guidelines; (a) the initial argument should be the output node
followed by the rest hyper-parameters of the method and (b) they must
provide a basic inference functionality, in most cases
$<$\textit{method}$>$\textit{.sample()} returning a predefined
\textit{elfi.Result} object containing the obtained samples along with
some other useful functionalities (e.g.\ plotting the marginal
posteriors).

The collection of likelihood-free inference methods implemented so far
contains the \textit{ABC Rejection Sampler} and \textit{Sequential
  Monte Carlo ABC Sampler}. A quite central method implemented by ELFI
is the \textit{Bayesian Optimisation for Likelihood-Free Inference
  (BOLFI)}, which is methodologically quite close to the ROMC method
that we implement in the current dissertation.

Add parallelisation ...



The Engine for Likelihood-Free Inference (ELFI) \cite{1708.00707} is a
Python software library dedicated to likelihood-free inference
(LFI). ELFI models in a convenient manner all the fundamental
components of a Probabilistic Model such as priors, simulators,
summaries and distances. Furthermore, ELFI already supports a range of
likelihood-free inference methods proposed in the last years.

\subsubsection{Modelling}
\label{sec:modelling}

ELFI models the Probabilistic Model as a Directed Acyclic Graph (DAG);
it implements this functionality based on the package NetworkX, which
is designed for creating general purpose graphs. Although not
restricted to that, in most cases the structure of a likelihood-free
model follows the pattern presented in figure~\ref{fig:elfi-model};
there are edges that connect the \textit{prior} distributions to the
simulator, the simulator is connected to the summary statistics which
is turn are connected to the distance. The distance is the output
node. Samples can be obtained from all nodes through sequential
sampling. The nodes that are defined as
\textit{elfi.Prior}\footnote{The \textit{elfi.Prior} functionality is
  a wrapper around the scipy.stats package.} are automatically
considered as the parameters of interest and they are the only nodes
that, apart from sampling, they should also provide pdf
evaluation. The function passed as argument in the
\textit{elfi.Summary} node can be any valid Python function with
arguments the prior variables. Finally, the observations should be
passed in the appropriate node through the argument
\pinline{observed}; all the nodes afterwards are evaluated at the
observations as well.

\begin{figure}[!ht]
    \begin{center}
      \includegraphics[width=0.8\textwidth]{./Thesis/images/chapter2/elfi.png}
    \end{center}
    \caption{Image taken from \cite{1708.00707}}
    \label{fig:elfi-model}
\end{figure}


\subsubsection{Inference Methods}
\label{sec:inference-methods}

The inference Methods implemented at the ELFI follow some common
guidelines; (a) the initial argument should be the output node
followed by the rest hyper-parameters of the method and (b) they must
provide a basic inference functionality, in most cases
$<$\textit{method}$>$\textit{.sample()} returning a predefined
\textit{elfi.Result} object containing the obtained samples along with
some other useful functionalities (e.g.\ plotting the marginal
posteriors).

The collection of likelihood-free inference methods implemented so far
contains the \textit{ABC Rejection Sampler} and \textit{Sequential
  Monte Carlo ABC Sampler}. A quite central method implemented by ELFI
is the \textit{Bayesian Optimisation for Likelihood-Free Inference
  (BOLFI)}, which is methodologically quite close to the ROMC method
that we implement in the current dissertation.

Add parallelisation ...




%%%%%%%%%%%%%%%%%%%%%%%%%%%%%%%%%%%%%%%% 
\clearpage
\section{Implementation}
In this chapter, we will exhibit the implementation of the ROMC
inference method at the ELFI package. The presentation is split into
two logical blocks; In
Sections~\ref{subsec:general_design},~\ref{subsec:training},~\ref{subsec:inference},
\ref{subsec:evaluation} we present the functionalities provided by our
implementation, from the \textbf{user's point-of-view}. These sections
demonstrate how a practitioner could use our ROMC implementation for
performing the inference in a real-case scenario. For providing a
practical overview of the implementation, we set-up a simple running
example and illustrate the functionalities on top of it. In contrast,
in the final Section~\ref{subsec:developers} we delve into the
internals of the code, presenting all the tiny details of the
implementation. This Section mainly refers to a \textbf{developer or a
  researcher} who would like to use ROMC as a meta-algorithm and
experiment with novel approaches for solving specific tasks. We have
designed our implementation preserving extensibility and
customisation; hence, a researcher may intervene in parts of the
method without too much effort. This Section can serve as a driver for
achieving so.


\subsection{General Design}
\label{subsec:general_design}
% \begin{figure}[!ht]
    \begin{center}
      \includegraphics[width=0.75\textwidth]{./Thesis/graphs/ROMC.png}
    \end{center}
    \caption{Overview of the ROMC implementation. The training part follows a sequential pattern; the functions in the green ellipses must be called in a sequential fashion for completing the training part and define the posterior distribution. The functions in blue ellipses are the API calls that are called by the user.}
    \label{fig:elfi-model}
\end{figure}
\begin{figure}[!ht]
    \begin{center}
      \includegraphics[width=0.75\textwidth]{./Thesis/graphs/ROMC.png}
    \end{center}
    \caption{Overview of the ROMC implementation. The training part follows a sequential pattern; the functions in the green ellipses must be called in a sequential fashion for completing the training part and define the posterior distribution. The functions in blue ellipses are the API calls that are called by the user.}
    \label{fig:elfi-model}
\end{figure}

\subsection{Training}
\label{subsec:training}
The training part has 4 functionalities; $romc.solve\_problems()$, $romc.estimate\_regions()$, $romc.fit\_posterior$.

\subsubsection*{Solve Problems}

\begin{lstlisting}[language=Python]
  romc.solve_problems(n1, seed=None, use\_bo=False)
\end{lstlisting}

\noindent
The functionality \lstinline{romc.solve_problems(n1, seed=None, use_bo=False)} is responsible for (a) drawing the nuisance variables, (b) define the optimisation problems and (c) solve them using either a gradient-based optimiser or Bayesian optimisation.



\subsection{Performing the Inference}
\label{subsec:inference}
The inference part provides 4 functionalities:

\begin{itemize}
\item \mintinline{python}{romc.sample(n2, seed=None)}
\item \mintinline{python}{romc.eval_unnorm_posterior(theta)}
\item \mintinline{python}{romc.eval_posterior(theta)}
\item \mintinline{python}{romc.compute_expectation(h)}
\end{itemize}


\subsubsection*{\mintinline{python}{romc.sample(n2)}}

This is the basic inference utility of the ROMC implementation. The samples are drawn from a uniform distribution $q_i$ defined over the corresponding bounding box and the weight $w_i$ is computed as in equation~\eqref{eq:sampling}.


\subsubsection*{Example - Sampling}

\begin{minted}
[framesep=2mm,
baselinestretch=1.2,
fontsize=\small,
]
{python}
'''Sampling part'''
n2 = 20
tmp = romc.sample(n2=n2, seed=seed)

# visualize region, adding the samples now
romc.visualize_region(i=1)

# Visualise marginal (built-in ELFI tool)
romc.result.plot_marginals(weights=romc.result.weights, bins=100, density=True, range=(-3,3))
plt.show(block=False)

# Summarize the samples (built-in ELFI tool)
romc.result.summary()
### Prints ###
# Number of samples: 1720
# Sample means: theta: -0.0792

# compute expecation
print("Expected value   : %.3f" % romc.compute_expectation(h = lambda x: np.squeeze(x)))
# Expected value   : -0.079

print("Expected variance: %.3f" % romc.compute_expectation(h =lambda x: np.squeeze(x)**2))
# Expected variance: 1.061

\end{minted}

\begin{figure}[h]
    \begin{center}
      \includegraphics[width=0.48\textwidth]{./Thesis/images/chapter3/example_marginal.png}
      \includegraphics[width=0.48\textwidth]{./Thesis/images/chapter3/example_region_samples.png}
    \end{center}
  \caption{Histogram of distances and visualisation of a specific region.}
  \label{fig:example_training}
\end{figure}


\subsubsection*{Example - Evaluate Posterior}

The \pinline{romc.eval_unnorm_posterior(theta)} evaluates the posterior at point $\theta$ using the expression \eqref{eq:approx_posterior}. The \pinline{romc.eval_posterior(theta)} approximates the partition function $Z = \int_{\thetab} p_{d,\epsilon}(\thetab|\data) d\thetab$ using the Riemann approximation in the points where the prior has mass; hence it doesn't scale well to high-dimensional spaces. In our simple example, this utility can provide a nice plot of the approximate posterior.

\begin{minted}
[framesep=2mm,
baselinestretch=1.2,
fontsize=\small,
]
{python}
'''Evaluate posterior'''
tmp = romc.sample(n2=n2, seed=seed)

\end{minted}

\begin{figure}[h]
    \begin{center}
      \includegraphics[width=0.75\textwidth]{./Thesis/images/chapter3/example_posterior.png}
    \end{center}
  \caption{Approximate posterior evaluation and histogram of the samples drawn.}
  \label{fig:example_posterior}
\end{figure}


\subsection{Evaluation}
\label{subsec:evaluation}
The ROMC implementation provides two functions for evaluating the inference results,

\begin{itemize}
\item \mintinline{python}{romc.compute_divergence(gt_posterior, step=0.1, distance="Jensen-Shannon")}
\item \mintinline{python}{romc.compute_ess()}
\end{itemize}

The \mintinline{python}{romc.compute_divergence(gt_posterior, step=0.1, distance="Jensen-Shannon")} 

\subsubsection*{Function (i): \mintinline{python}{romc.compute_divergence(theta)}}

This function computes the divergence between the ROMC approximation
and the ground truth posterior. Since the compuation of the divergence
is performed using the Riemann approximation in can only work in low
dimensional parametric spaces; it is suggested to be used for up to a
$3D$ parametric space.As mentioned in the beginning of this chapter,
in a real-case scenario it is not expected the ground-truth posterior
to be available; this is the whole meaning of performing the
inference. However, there are two scenarios where this functionality
can be useful; (a) when the likelihood is tractable and we want to
check the accuracy of the ROMC method (as in the running-example) (b)
when we want to compute the divergence between the ROMC's posterior
approximation and another approximation e.g.\ ABC Rejection. The
argument \pinline{step} defines step used in the Riemann approximation
and the argument \pinline{distance} can take either the
\pinline{Jensen-Shannon} or the \pinline{KL-divergence} value, for
computing the appropriate distance.

\subsubsection*{Function (ii): \mintinline{python}{romc.compute_ess()}}

This function compute the Effective Sample Size (ESS) using the following expression,

\begin{equation} \label{eq:ESS}
  ESS = \frac{(\sum_i w_i)^2}{\sum_i w_i^2}
\end{equation}


\begin{pythoncode}
res = romc.compute_divergence(wrapper, distance="Jensen-Shannon")                                 
print("Jensen-Shannon divergence: %.3f" % res)
# Jensen-Shannon divergence: 0.025

print("Nof Samples: %d, ESS: %.3f" % (len(romc.result.weights), romc.compute_ess()))
# Nof Samples: 19950, ESS: 16694.816
\end{pythoncode}

\subsection{Implementation details for developers}
\label{subsec:developers}
\begin{figure}[h]
    \begin{center}
      \includegraphics[width=0.95\textwidth]{./Thesis/graphs/RomcEntityDiagram.png}
    \end{center}
  \caption{Histogram of distances and visualisation of a specific region.}
  \label{fig:example_training}
\end{figure}
lala

\clearpage
\section{Experiments}

\subsection{Another Example}

Here I will probably add a simple 2D-example (MA2 is a good option).

\subsection{Another Example}

Here I will probably add an example withoud gradients, for showing the
Bayesian Optimisation approach.

\subsection{Execution Time Experiments}

Here I will provide some graphs showing how the inference scales relatively to some hypereparameters (i.e. $n_1, n_2$).

If time allows, I will also add the timings from a parallelised implementation.

\section{Conclusions}

\subsection{Outcomes}

\subsection{Future Research Directions}
\clearpage

\printbibliography
\clearpage

\appendix
\section*{Appendices}
\addcontentsline{toc}{section}{Appendices}

\clearpage
\section{An Appendix}
\label{app:one}

Some stuff.
\clearpage

\section{Another Appendix}
\label{app:two}

Some other stuff.



\end{document}
