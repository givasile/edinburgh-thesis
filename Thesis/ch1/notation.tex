\subsection{Notation}
\label{subsec:notation}

In this section, we keep an overview of the symbols used throughout
the document along with their explanation, in physical language. Some
quantities that are discribed informally here will be defined formally
in the next chapters. We try to keep the notation as consistent as
possible throughout the document.

 Some more generic symbols. In general, as $p(\cdot)$ we represent any valid pdf and $p(\cdot | \cdot)$ any conditional distribution. Depending on the content and the symbol used at the placeholder the distribution may have a more specific meaning.


\subsubsection*{Random Generator:}
\label{sec:random-generator}
\begin{itemize}
\item $M_r(\thetab): \R^D \rightarrow \R$
\end{itemize}

\subsubsection*{Variables}
\label{sec:variables}

\begin{itemize}
\item $D$: the dimensionality of the parameter-space
\item $\thetab \in \R^D$: the parameters of interest
\item $\data$: the observations
\item $\epsilon$: threshold
\item $\vb \in \R^N$, random variable that adds the stochasticity to the generator. 
\item $\vb_i \sim \vb$: a sample drawn from $\vb$
\item $\Y_{\thetab}$ random variable descriping the simulator $M_r(\thetab)$. The pdf of $\Y_{\thetab}$ is unknown in closed form or intractable to be evaluated.
\item $\yb_i \sim \Y_\theta$ a sample drawn from $\Y_\theta$. The sample can be obtained by executing the simulator $\yb_i \sim M_r(\thetab)$
\end{itemize}


\subsubsection*{Sets}
\label{sec:sets}

\begin{itemize}
\item $\region(\data)$ the set of points $\yb$ around the observations $\data$, i.e. $\yb := \{\yb: d(\yb, \data) < \epsilon \}$
\item $\regioni = \region(\yb_i)$ the set of points $\yb$ around $\yb_i$, i.e. $\yb := \{\yb: d(\yb, \yb_i) < \epsilon \}$
\end{itemize}
    
\subsubsection*{Generic Functions}
\label{sec:generic-functions}

\begin{itemize}
\item $p(\thetab)$: the prior distribution
\item $p(\thetab|\data)$: the posterior distribution
\item $p_{d,\epsilon}(\thetab|\data)$: the approximate posterior distribution
\item $d(\mathbf{x}, \mathbf{y}): \R^{2N} \rightarrow \R$: any valid distance, e.g L2 norm: $||\mathbf{x}-\mathbf{y}||_2^2$
\end{itemize}

\subsubsection*{Functions (Mappings):}
\label{sec:functions-mappings}

\begin{itemize}
\item $M_d(\thetab, \vb): \R^D \rightarrow \R$ the deterministic generator; if we pass the state $v$ of all stochastic variables that are part of the data generation process, then producing an outcome becomes deterministic.
\item $f_i(\thetab) = M_d(\thetab, \vb_i)$ alias

\item $g_i(\thetab) = d(f_i(\thetab), \data)$
\item $T(\mathbf{x}): \mathbb{R}^{D_1} \rightarrow \mathbb{R}^{D_2}$ where $D_1 > D_2$, the summary statistic mapping.
\item  $\indicator{\region(\data)}(\yb)$ the indicator function:
    \begin{gather*}\indicator{\region(\data)}(\yb) = \left\{
	\begin{array}{ll}
		1 & \mbox{if } d(\yb,\data) \leq \epsilon \\
		0 & \mbox{else } 
	\end{array} \right. \end{gather*}

\item $L(\theta)$ the likelihood function
\item $L_{d,\epsilon}(\theta)$ the approximate likelihood function
\end{itemize}   
