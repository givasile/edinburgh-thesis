\subsection{Notation}
\label{subsec:notation}

In this section, we present an overview of the symbols and the
quantities used in the document. At this point, the quantities are
described quite informally. Most of them will be defined formally in
the next chapters. We try to keep the notation as consistent as
possible throughout the document. The symbol $\R^N$, when used, describes 
that a variable belongs to a multi-dimensional space in $\R$; $N$
doesn't represent a specific number.

\subsubsection*{Random Generator}
\label{sec:random-generator}
\begin{itemize}
\item $M_r(\thetab): \R^D \rightarrow \R$: The black-box data simulator.
\end{itemize}

\subsubsection*{Parameters/Random Variables/Symbols}
\label{sec:variables}

\begin{itemize}
\item $D \in \R$, the dimensionality of the parameter-space
\item $\thetab \in \R^D$, the parameters of interest
\item $\data n\R^N$, the observations
\item $\epsilon \in R$, the threshold for defining the region around $\data$
\item $\vb \in \R^N$, random variable that represents the stochasticity of
  the generator.
\item $\vb_i \sim \vb$, a specific sample drawn from $\vb$
\item $\Y_{\thetab}$, random variable descriping the simulator
  $M_r(\thetab)$. The pdf of $\Y_{\thetab}$ is either not unknown or intractable
\item $\yb_i \sim \Y_\theta$, a sample drawn from $\Y_\theta$. The
  sample is obtained by executing the random simulator
  $\yb_i \sim M_r(\thetab)$
\end{itemize}


\subsubsection*{Sets}
\label{sec:sets}

\begin{itemize}
\item $\region(\data)$, the set of points $\yb$ around the observations
  $\data$, i.e. $\yb := \{\yb: d(\yb, \data) < \epsilon \}$
\item $\regioni = \region(\yb_i)$, the set of points $\yb$ around
  $\yb_i$, i.e. $\yb := \{\yb: d(\yb, \yb_i) < \epsilon \}$
\end{itemize}
    
\subsubsection*{Generic Functions}
\label{sec:generic-functions}

\begin{itemize}
\item $p(\cdot)$, any valid pdf
\item $p(\cdot | \cdot)$, any valid conditional distribution.
\item $p(\thetab)$, the prior distribution
\item $p(\thetab|\data)$, the posterior distribution
\item $p_{d,\epsilon}(\thetab|\data)$, the approximate posterior
  distribution  
\item $d(\mathbf{x}, \mathbf{y}): \R^{2N} \rightarrow \R$: any valid
  distance e.g L2 norm: $||\mathbf{x}-\mathbf{y}||_2^2$
\end{itemize}

\subsubsection*{Functions (Mappings)}
\label{sec:functions-mappings}

\begin{itemize}
\item $M_d(\thetab, \vb): \R^D \rightarrow \R$, the deterministic
  generator; representing all stochastic variables that
  are part of the data generation process with the \textbb{parameter} $\vb$, transform the stochastic generator to a deterministic.
\item $f_i(\thetab) = M_d(\thetab, \vb_i)$, deterministic generator associated with parameter $\vb_i$
\item $g_i(\thetab) = d(f_i(\thetab), \data)$, distance of the generated data from the observations
\item $T(\mathbf{x}): \mathbb{R}^{D_1} \rightarrow \mathbb{R}^{D_2}$
  where $D_1 > D_2$, the summary statistic mapping.
\item $\indicator{\region(\data)}(\yb)$, the indicator function; returns 1 iff $d(\yb, \data)<\epsilon$
\item $L(\thetab)$, the likelihood
\item $L_{d,\epsilon}(\thetab)$, the approximate likelihood
\end{itemize}
