In this section, we present an overview of the symbols that will be
used in the rest of the document. At this level, the quantities are
introduced quite informally, through general descriptions. Most of
them will be defined formally in the next chapters. We try to keep the
notation as consistent as possible throughout the document. The symbol
$\R^N$, when used, describes that a variable belongs to the 
$N-dimensional$ euclidean space; $N$ doesn't represent a specific
number.

\subsubsection*{Random Generator}
\label{sec:random-generator}
\begin{itemize}
\item $M_r(\thetab): \R^D \rightarrow \R$: The black-box data simulator.
\end{itemize}

\subsubsection*{Parameters/Random Variables/Symbols}
\label{sec:variables}

\begin{itemize}
\item $D \in \mathbb{N}$, the dimensionality of the parameter-space
\item $\Thetab \in \R^D$, random variable representing the parameters of interest
\item $\data \in \R^N$, the vector of observations
\item $\epsilon \in R$, the threshold setting the limit on the region around $\data$
\item $\V \in \R^N$, random variable representing the stochasticity of
  the generator. It is also called nuisance variable, because we are not interested in infering a postrior distribution on it.
\item $\vb_i \sim \V$, a specific sample drawn from $\V$
\item $\Y_{\thetab}$, random variable descriping the simulator $M_r(\thetab)$. 
\item $\yb_i \sim \Y_\theta$, a sample drawn from $\Y_\theta$. It can
  be obtained by executing the simulator $\yb_i \sim M_r(\thetab)$
\end{itemize}


\subsubsection*{Sets}
\label{sec:sets}

\begin{itemize}
\item $\region(\data)$, the set of points $\yb := \{\yb: d(\yb, \data) \leq \epsilon \}$
\item $\regioni$, the set of points defined around $\yb_i$ i.e.\ $\regioni = \region(\yb_i)$
\item $S_i$, the set of $\thetab$ points, such that $\{ \thetab \in \region(\yb_i=M_d(\thetab, \vb_i)) \}$
\end{itemize}
    
\subsubsection*{Generic Functions}
\label{sec:generic-functions}

\begin{itemize}
\item $p(\cdot)$, any valid pdf
\item $p(\cdot | \cdot)$, any valid conditional distribution.
\item $p(\thetab)$, the prior distribution on the parameters
\item $p(\vb)$, the prior distribution on the nuisance variables
\item $p(\thetab|\data)$, the posterior distribution
\item $p_{d,\epsilon}(\thetab|\data)$, the approximate posterior
  distribution  
\item $d(\mathbf{x}, \mathbf{y}): \R^{2N} \rightarrow \R$: any valid
  distance, the $L_2$ norm: $||\mathbf{x}-\mathbf{y}||_2$
\end{itemize}

\subsubsection*{Functions (Mappings)}
\label{sec:functions-mappings}

\begin{itemize}
\item $M_d(\thetab, \vb): \R^D \rightarrow \R$, the deterministic
  generator; all stochastic variables that are part of the data generation process are represented by the \textbf{parameter} $\vb$
\item $f_i(\thetab) = M_d(\thetab, \vb_i)$, deterministic generator associated with sample $\vb_i \sim p(\vb)$
\item $g_i(\thetab) = d(f_i(\thetab), \data)$, distance of the generated data $f_i(\thetab)$ from the observations
\item $T(\mathbf{x}): \mathbb{R}^{D_1} \rightarrow \mathbb{R}^{D_2}$
  where $D_1 > D_2$, the mapping that computes the summary statistic
\item $\indicator{\region(\data)}(\yb)$, the indicator function; returns 1 if $d(\yb, \data) \leq \epsilon$, else 0
\item $L(\thetab)$, the likelihood
\item $L_{d,\epsilon}(\thetab)$, the approximate likelihood
\end{itemize}
