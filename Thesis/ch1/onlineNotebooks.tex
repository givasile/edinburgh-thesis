Since the dissertation is mainly focused on the implementation side of
the ROMC inference approach, we have created some jupyter Notebooks
for supporting the pdf document. The notebooks offer a wide range of
features that would be impossible to be presented in a static pdf
format. Through the notebooks the reader can (a) check the complete
code for performing the inference end-to-end (b) understand the
functionalities of our implementation (c) check the validity of our
claims and (c) can interactively execute experiments.

The notebooks are provided in two different formats:\footnote{The
  notebooks in the two repositories are identical.}

\begin{itemize}
\item In the github repository
  \href{https://github.com/givasile/edinburgh-thesis/tree/master/notebook_examples}{https://github.com/givasile/edinburgh-thesis/tree/master/notebook_examples}. For
  playing around with the examples, the reader should \textit{clone} the
  repository locally, execute the installation instructions and run the
  notebooks using the \textit{jupyter} package.
\item As stand-alone google colab notebooks. Using this format, the
  reader can follow the link, view the notebook and make an online
  clone of it. Hence, without any installation overhead, he/she may
  interactivelly explore all the provided functionalities.
\end{itemize}


Throughout the document, we may refer to the notebooks when we want to
let the reader observe a specific functionality that we describe. The
following list contains the implemented examples along with their
links:

\begin{itemize}
\item Simple 1D example
  \begin{itemize}
  \item \href{https://colab.research.google.com/drive/13znoPzew3sS89QD-j5usEOefIPNu655Y?usp=sharing}{Google colab}
  \item \href{https://github.com/givasile/edinburgh-thesis/blob/master/notebook_examples/example_1D.ipynb}{Github repository}
  \end{itemize}
 
\item Simple 2D example
  \begin{itemize}
  \item
    \href{https://colab.research.google.com/drive/1T8919FCAi2w9MXm9XKT_iJLnB0y1EN32?usp=sharing}{Google colab}
  \item \href{https://github.com/givasile/edinburgh-thesis/blob/master/notebook_examples/example_2D.ipynb}{Github repository}
  \end{itemize}

\item Moving Average example  
  \begin{itemize}
  \item
    \href{https://colab.research.google.com/drive/145s0-sMSUgi30MeS48SmiJlkd5GYrSQU?usp=sharing}{Google colab}
  \item
    \href{https://github.com/givasile/edinburgh-thesis/blob/master/notebook_examples/example_ma2.ipynb}{Github repository}
  \end{itemize}

\item Extensibility example
  \begin{itemize}
  \item
    \href{https://colab.research.google.com/drive/145s0-sMSUgi30MeS48SmiJlkd5GYrSQU?usp=sharing}{Google colab}
  \item
    \href{https://github.com/givasile/edinburgh-thesis/blob/master/notebook_examples/example_MA2.ipynb}{Github repository}
  \end{itemize}

\end{itemize}

All notebooks provide a practical overview of the implemented
functionalities, and it is effortless to use them; in particular, the
google colab version is entirely plug-and-play. Therefore, we
encourage the reader of the dissertation to use them as supporting
material.
