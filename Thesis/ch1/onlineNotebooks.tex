Since the dissertation is mainly focused on the implementation of the
ROMC, we have created some interactive notebooks\footnote{We have used
  the jupyter notebooks \autocite{Kluyver:2016aa}} supporting the main
document. The reader is advised to exploit the notebooks in order to
(a) review the code used for performing the inference (b) understand
the functionalities of our implementation in a practical sense (c)
check the validity of our claims and (c) interactively execute
experiments.

We use google colab notebooks \autocite{Bisong2019} to relieve the reader from the overhead
of creating a local environment to run the examples. Instead, they may just open the notebooks in the following links and play with the examples.

\begin{itemize}
\item \href{https://colab.research.google.com/drive/1lGRp0XrNfZ64NN0ASB_tYEKowXwlveDC}{Simple 1D example}
\item \href{https://colab.research.google.com/drive/1Fof_WmCi1YizzSI_63aEsbLXsno5gSZ3}{Simple 2D example}
\item \href{https://colab.research.google.com/drive/1nkdACQ370SSc0KB1bHv4sBRaxMlMqoNH}{Moving Average example}
\item \href{https://colab.research.google.com/drive/1RzB-V1QueP1y1nyzv_VOqR1nVz3DUH3v}{Tutorial for extending the ROMC method}
\end{itemize}
