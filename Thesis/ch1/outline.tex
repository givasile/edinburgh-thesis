The remainder of the dissertation is organised as follows. In Chapter 2, we establish the mathematical formulation; specifically, we
initially describe the simulator-based models and provide some
fundamental algorithms that have been proposed for performing
inference in these set-ups. Afterwards, we provide the mathematical
description of the ROMC approach \autocite{Ikonomov2019}. Finally, we depict the mathematical description into an algorithmic view. In Chapter 3, we illustrate the implementation part; we initially provide some
information regarding the Python package Engine for Likelihood-Free
Inference (ELFI) \autocite{1708.00707} and subsequently, we present the
implementation details of ROMC in this package. In general, the logical connectivity of the dissertation unit Chapter 3 follows the scheme; Mathematical modelling $\rightarrow$ Algorithm $\rightarrow$ Software.

In Chapter 4, we demonstrate the functionalities of the ROMC implementation at some
real-world examples; this chapter desires to demonstrate the success of
the ROMC method and our implementation's at Likelihood-Free
tasks. Finally, in Chapter 5, we conclude with some thoughts on the
work we have done and some future research ideas.