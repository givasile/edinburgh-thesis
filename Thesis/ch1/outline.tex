The remainder of the dissertation is organised as follows; in Chapter
2, we establish the mathematical formulation. We initially describe
the simulator-based models and provide some background information on
the fundamental algorithms proposed so far. Afterwards, we provide the
mathematical description of the ROMC approach
\autocite{Ikonomov2019}. Finally, we depict the mathematical
description into an algorithmic view. In Chapter 3, we illustrate the
implementation part; we initially provide some information regarding
the Python package Engine for Likelihood-Free Inference (ELFI)
\autocite{1708.00707} and subsequently, we present the implementation
details of ROMC in this package. In general, the conceptual scheme
followed by the dissertation is Mathematical modelling $\rightarrow$
Algorithm $\rightarrow$ Software.

In Chapter 4, we demonstrate the functionalities of the ROMC
implementation at some real-world examples; this chapter demonstrates
the success of the ROMC method and our implementation's at
likelihood-free tasks. Finally, in Chapter 5, we conclude with some
thoughts on the work we have done and some future research ideas.