In this dissertation, we have studied the likelihood-free inference approaches, focusing on the novel ROMC method that we implemented in an open-source software package. The main contribution of the dissertation is the implementation mentioned above which can be used mainly from the research community for further experimentation.

In chapter 2, we presented the simulator-based models explaining the particularities of the inference when the likelihood is not tractable. We then presented an overview of the Optimisation Monte-Carlo methods (OMC and ROMC) examining their strategies on approximating the posterior. We discussed the point-of-view of the ROMC approach, demonstrating the mathematic modelling it introduces. We also presented the aspect of ROMC as a meta-algorithm. Finally, at the end of chapter 2, we transformed the mathematical modelling of ROMC to algorithmic form. Up to this point, the dissertation mainly restates the ideas presented in the original paper ... and, hence, it can be used supplementarily with the paper by a reader who wants to understand the ROMC approach.

The notable contribution of the dissertation is the implementation of the method in the ELFI package. Due to the novelty of the method, it has not been implemented in any other package so far. We tried to implement the method focusing on four principles; simplicity, accuracy, efficiency and extensibility. We tried to provide the user with a simple-to-use method. Therefore, we followed the guidelines of the ELFI package, aligning ROMC with the rest inference methods. We also kept the function calls simple, asking only for the necessary arguments. Secondly, we tested the implementation in a range of examples for ensuring accurate inference results. We designed some artificial examples for evaluating our implementation using ground-truth information. We also tested that it works smoothly under general models, not artificially created by us. Thirdly, we tried to solve the tasks efficiently. We avoided using redundant calls, we exploited vectorisation when possible, and we avoided expensive for-loops.  We measured the execution time of the method for providing an overview of the time needed for performing each task. Finally, we preserved extensibility; this was a significant priority in the implementation design. Apart from offering a ready-to-use inference method, we want our implementation to serve as the initial point for researchers who would like to further experiment with the method. This requirement aligns with the nature of ROMC as a meta-algorithm; one can replace the method involved in a specific task without the rest of the algorithm to collapse. Finally, we provide extensive documentation of all implemented methods and a  collection of examples for illustrating the main use-cases. We wanted the reader to be able to observe and interact in-practice with the functionalities we offer.