In this section, we attempt the depiction of the mathematical
description of ROMC in algorithms. Specificaly, in chapter
\ref{subsubsec:romc-meta-algorithm} we present the general algorithmic
description of ROMC as a meta-algorithm and in chapter
\ref{subsubsec:alg-training-inference} the proposals of ROMC for
solving the training and the inference parts.

\subsubsection{ROMC as a Meta-Algorithm}
\label{subsubsec:romc-meta-algorithm}

As stated at the introductory chapter, ROMC can be understood as
step-by-step alorithmic approach for perfroming the inference in
Simulator-Based Models. The particular methods used for solving the
sub-tasks are left as a free choice to the user. As presented in
Algorithm~\ref{alg:meta_alg}, the methods involved in solving the
optimistation problem (step~\ref{algstep:optimise}) and constructing
the bounding box (step~\ref{algstep:bounding_box}) are not
restricted. The practiosioner may choose any convenient algorithm,
judging the trade-offs between accuracy, robustness, efficiency and
complexity. In particular for the optimisation step, the choice of the
appropriate optimiser should also consider the properties of
$g_i(\thetab)$. Some important questions that should be considered are
whether the function differentiable and if so whether we know the
gradients $\nabla_{\thetab} [g_i] $ in closed-form. As described in
sections~\ref{subsubsec:GB_approach} and~\ref{subsubsec:GP_approach},
ROMC proposes two alternative optimisation schemes (gradient-based and
gaussian-process approach) depending on whether the gradients are
available or not.

\begin{algorithm}[t]
	\caption{ROMC as a Meta-Algorithm. Requires $M_r(\theta), y_0$. Hyperparameters $n_1,n_2$.}\label{alg:meta_alg}
	\begin{algorithmic}[1]
		\For{$i \gets 1 \textrm{ to } n_1$}
    \State Sample a random state $\vb_i \sim p(\vb)$
		\State Define the deterministic mapping $f_i(\thetab) = M_d(\thetab, \vb)$ and therefore $g_i(\thetab) = d(f_i(\theta), y_0)$.
    \State Obtain $d_i^* = \text{min}_{\thetab} \: [g_i(\thetab)]$ and $\thetab_i^* = \text{argmin}_{\thetab}\: [g_i(\thetab)]$ using any convenient optimiser. \label{algstep:optimise}
    \State Approximate the local area $\{ \thetab : g_i(\thetab) < \epsilon$ and $d(\thetab, \thetab_i^*) < M \}$ with a Bounding Box, using any convenient method. \label{algstep:bounding_box}
		\State Define a uniform distribution $q_i(\thetab)$ over the Bounding Box.
			\For{$j \gets 1 \textrm{ to } n_2$}
			\State $\thetabij \sim q_i(\thetab)$
			\State Accept $\thetabij$ as posterior sample with weight $w_{ij} = \frac{p(\thetabij)}{q_i(\thetabij)} \indicator{\regioni} (\thetabij)$
			\EndFor
      \EndFor
     \Return(List with samples $\thetabij$ and weights $w_{ij}$) 
	\end{algorithmic}
\end{algorithm}


\subsubsection{Training and Inference Algorithms}
\label{subsubsec:alg-training-inference}

In this section, we will provide the algorithmic description of the
ROMC method; (a) the procedures for solving the optimisation problems
using either the gradient based approach or the Gaussian Process
alternative and (b) the construction of the Bounding Box. Afterwards,
we will discuss the advantages and the disadvantages of each choice
both in terms of accuracy and efficiency.

\noindent
At a high-level, the ROMC method can be split into the training and
the inference part.

\noindent
At the training (fitting) part, the goal is the estimation of the
proposal regions $q_i$. The steps include (a) sampling the nuisance
variables $\vb_i \sim p(\vb)$ (b) defining the optimisation problems
$\min_{\thetab} [g_i(\thetab)]$ (c) obtaining $\thetab_i^*$ (d) checking
whether $d_i^* < \epsilon$ and (e) building the bounding box for
obtaining the proposal region $q_i$. If gradients are available, using
a gradient-based method is adviced for obtaining $\thetab_i^*$ much
faster.  Providing $\nabla_{\thetab} g_i$ in closed-form provides an
upgrade in both accuracy and efficiency; If closed-form description is
not available, approximate gradients with finite-differences
$\frac{\partial g_i(\thetab)}{\thetab_d} = \frac{g_i(\thetab_d + h
  \mathbf{e_d}) - g_i(\thetab_d)}{h}$ requires two evaluations of $g_i$ for
\textbf{every} parameter $\thetab_d$. For low-dimensional problems
though, this approach still works well. When gradients are not
available or $g_i$ is not differentiable, using the Gaussian Process
is the only solution. In this case, the training part is much slower
due to the fitting of the surrogate model and the ignorance of the
slope throughout the optimisation procedure. Nevertheless, computing
the proposal region $q_i$ becomes faster since $\hat{d}_i$ can be used
instead of $g_i$ which involves running the whole simulator
$M_d(\thetab, \vb_i)$ for each query. The algorithms are presented
in~\ref{alg:training_GB} and~\ref{alg:training_GP}.

\noindent
Performing the inference includes (a) evaluating the unnormalised
posterior $p_{d, \epsilon}(\theta_b|\data)$ (b) sampling from the
posterior $ \thetab_i \sim p_{d, \epsilon}(\theta_b|\data)$ and (c)
computing an expectation $E_{\thetab|\data}[h(\thetab)]$.  Computing
an expectation can be done easily after weighted samples are obtained
\ref{eq:expectation}, so we will not discuss it seperately.

\noindent
For evaluating the unnormalized posterior in the gradient-based
approach, only the deterministic functions $g_i$ and the prior
distribution $p(\thetab)$ are required; there is no need for solving
the optimisation problems and building the proposal regions. The
evaluation requires iterating over all $g_i$ and evaluating the
distance from the observed data. In contrast, using the GP approach,
the optimisation part should be performed first for fitting the
surrogate models $\hat{d}_i(\thetab)$ and evaluate the indicator
function on them. This provides an important speed-up, especially when
running the simulator is computationally expensive. The evaluation of
the posterior is presented analytically in~\ref{alg:posterior_GB}
and~\ref{alg:posterior_GP}.

\noindent
Sampling is performed by getting $n_2$ samples from each proposal
region $q_i$. For each sample $\thetab_{ij}$, the indicator function
is evaluated $\indicator{\regioni(\data)}(\thetab_{ij})$ for checking
if it lies inside the acceptance region. If so the corresponding
weight is computed as in \cite{eq:sampling}. As before, if a surrogate
model $\hat{d}$ has been fitted, it can be used for the evaluation of
the indicator function providing again a speedup. Apparently, the
compuational benefit is more important compared to posterior
evaluation, because the indicator function must be evaluated for a
total of $n_1 \times n_2$ points. The sampling algorithms are
presented step-by-step in~\ref{alg:sampling_GB}
and~\ref{alg:sampling_GP}.

\noindent
As a conclusion, we can state that the choise of using a bayesian
optimisation approach provides a significant speed-up in the inference
part with the cost of making the training part slower and a possible
approximation error. It is typical in many Machine-Learning use cases,
being able to provide enough time and computational resources for the
training phase, but asking for efficiency in the inference
part. Having that in mind, we can say that the Gaussian-Process is a
quite usefull alternative.

\begin{minipage}{0.46\textwidth}
\begin{algorithm}[H]
    \centering
    \caption{Training Part - Gradient approach. Requires $g_i(\theta), p(\theta)$}\label{alg:training_GB}
    \begin{algorithmic}[1]
      \For{$i \gets 1 \textrm{ to } n$}
        \State Obtain $\theta_i^*$ using a Gradient Optimiser
        \If{$g_i(\theta_i^*) > \epsilon$}
        \State{go to} 1
        \Else
        \State Approximate $H_i \approx J^T_iJ_i$
        \State Use algorihm~\ref{alg:region_construction} to obtain $q_i$
        \EndIf      
      \EndFor
      \Return{$q_i, p(\theta), g_i(\theta)$}
    \end{algorithmic}
\end{algorithm}
\end{minipage}
\hfill
\begin{minipage}{0.46\textwidth}
\begin{algorithm}[H]
    \centering
    \caption{Training Part - GP approach. Requires $g_i(\theta), p(\theta)$}\label{alg:training_GP}
    \begin{algorithmic}[1]
      \For{$i \gets 1 \textrm{ to } n$}
        \State Obtain $\theta_i^*, \hat{d}_i(\theta)$ using a GP approach
        \If{$g_i(\theta_i^*) > \epsilon$}
        \State{go to} 1
        \Else
        \State Approximate $H_i \approx J^T_iJ_i$
        \State Use algorihm~\ref{alg:region_construction} to obtain $q_i$
        \EndIf      
      \EndFor
      \Return{$q_i, p(\theta), \hat{d}_i(\theta)$}
    \end{algorithmic}
\end{algorithm}
\end{minipage}

\begin{algorithm}[!ht]
	\caption{Proposal Region $q_i$ construction; Needs, a model of distance $d$ ($\hat{d}$ or $g_i$), optimal point $\theta_i^*$, number of refinements $K$, step size $\eta$ and curvature matrix $\hessian_i$ ($J_i^TJ_i $ or GP Hessian)}\label{alg:region_construction}
	\begin{algorithmic}[1]
	\State Compute eigenvectors $\mathbf{v}_{d}$ of $H_i$ {\scriptsize ($d = 1,\ldots,||\theta ||)$}
	\For{$d \gets 1 \textrm{ to } ||\theta||$}
		\State $\Tilde{\theta} \gets \theta_i^*$ \label{algstep:box_constr_start}
		\State $k \gets 0$
		\Repeat
        	\Repeat
                \State $\Tilde{\theta} \gets \Tilde{\theta} + \eta \ \mathbf{v}_{d}$ \Comment{Large step size $\eta$.}
        	\Until{$d( (\Tilde{\theta}, i), ) \ge \epsilon$}
        	\State $\Tilde{\theta} \gets \Tilde{\theta} - \eta \ \mathbf{v}_{d}$
        	\State $\eta \gets \eta/2$ \Comment{More accurate region boundary}
        	\State $k \gets k + 1$
    	\Until $k = K$
    	\State Set final $\Tilde{\theta}$ as region end point. \label{algstep:box_constr_end}
    	\State Repeat steps~\ref{algstep:box_constr_start}~-~\ref{algstep:box_constr_end} for $\mathbf{v}_{d} = - \mathbf{v}_{d}$
	\EndFor
	\State Fit a rectangular box around the region end points and define $q_i$ as uniform distribution
	\end{algorithmic}
\end{algorithm}

\begin{minipage}{0.46\textwidth}
\begin{algorithm}[H]
    \centering
    \caption{Evaluate unnormalised posterior - Gradient approach. Requires $g_i(\theta), p(\theta)$}\label{alg:posterior_GB}
    \begin{algorithmic}[1]
      \State $k \leftarrow 0$
        \For {$i \gets 1 \textrm{ to } n_1$}
          \If {$g_i(\theta) > \epsilon$}
            \State $k \leftarrow k + 1$
          \EndIf
          \EndFor
      \Return{$kp(\theta)$}
    \end{algorithmic}
\end{algorithm}
\end{minipage}
\hfill
\begin{minipage}{0.46\textwidth}
\begin{algorithm}[H]
    \centering
    \caption{Evaluate unnormalised posterior - GP approach. Requires $\hat{d}_i(\theta), p(\theta)$}\label{alg:posterior_GP}
    \begin{algorithmic}[1]
      \State $k \leftarrow 0$
        \For {$i \gets 1 \textrm{ to } n_1$}
          \If {$d_i(\theta) > \epsilon$}
            \State $k \leftarrow k + 1$
          \EndIf
          \EndFor
      \Return{$kp(\theta)$}
    \end{algorithmic}
\end{algorithm}
\end{minipage}


\begin{minipage}{0.46\textwidth}
\begin{algorithm}[H]
    \centering
    \caption{Sampling - Gradient Based approach. Requires $g_i(\theta), p(\theta), q_i$}\label{alg:sampling_GB}
    \begin{algorithmic}[1]
      \For {$i \gets 1 \textrm{ to } n_1$}
      \For {$j \gets 1 \textrm{ to } n_2$}
          \State $\theta_{ij} \sim q_i$
          \If {$g_i(\theta_{ij}) > \epsilon$}
            \State Reject $\theta_{ij}$
          \Else {}
            \State $w_{ij} = \frac{p(\theta_{ij})}{q(\theta_{ij})}$
            \State Accept $\theta_{ij}$, with weight $w_{ij}$
          \EndIf
      \EndFor
      \EndFor
    \end{algorithmic}
\end{algorithm}
\end{minipage}
\hfill
\begin{minipage}{0.46\textwidth}
\begin{algorithm}[H]
    \centering
    \caption{Sampling - GP approach. Requires $\hat{d}_i(\theta), p(\theta), q_i$}\label{alg:sampling_GP}
    \begin{algorithmic}[1]
      \For {$i \gets 1 \textrm{ to } n_1$}
      \For {$j \gets 1 \textrm{ to } n_2$}
          \State $\theta_{ij} \sim q_i$
          \If {$\hat{d}_i(\theta_{ij}) > \epsilon$}
            \State Reject $\theta_{ij}$
          \Else {}
            \State $w_{ij} = \frac{p(\theta_{ij})}{q(\theta_{ij})}$
            \State Accept $\theta_{ij}$, with weight $w_{ij}$
          \EndIf
      \EndFor
      \EndFor
    \end{algorithmic}
\end{algorithm}
\end{minipage}
