\subsection{Simulator-Based (Implicit) Models}

As already stated at chapter \ref{sec:introduction}, in
Simulator-Based models we cannot evaluate the posterior
$p(\thetab|\data) \propto L(\thetab)p(\thetab)$, due to the
intractability of the likelihood $L(\thetab) = p(\data|\thetab)$. The
following equation allows incorporating the simulator in the place of
the likelihood and forms the basis of all Likelihood-Free inference
approaches.

\begin{equation} \label{eq:likelihood} L(\thetab) = \lim_{\epsilon \to
    0} c_\epsilon \int_{\yb \in B_\epsilon(\data)} p(\yb|\thetab)d\yb
  = \lim_{\epsilon \to 0} c_\epsilon Pr(M_r(\thetab) \in
  B_\epsilon(\data))
\end{equation}

\subsubsection{Approximate Bayesian Computation (ABC) Rejection
  Sampling}

ABC Rejection Sampling is a modified version of the traditional
Rejection Sampling method, for cases when likelihood evaluation is
intractable. In typical Rejection Sampling, a sample is obtained from
the prior $\thetab \sim p(\thetab)$ and it is maintained with
probability $L(\thetab)/ \text{max}_{\thetab} L(\thetab)$. Although we
cannot use this approach out of the box (evaluating $L(\thetab)$ is
impossible in our case), we can take modify the approach for using the
simulator.

In the discrete case scenario where $\Y_{\thetab}$ can take a finite
set of values, the likelihood becomes
$L(\thetab) = Pr(\Y_{\thetab} = \data)$ and the posterior
$p(\thetab|\data) \propto Pr(\Y_{\thetab}=\data)p(\thetab)$. We can
sample from the prior $\thetab_i \sim p(\thetab)$, run the simulator
$\yb_i = M_r(\thetab_i)$ and maintain $\thetab_i$ only if
$\yb_i = \data$.

The method above becomes less helpfull as the finite set of
$\Y_{\thetab}$ values grows larger, since the probability of
maintaining a sample (acceptance rate) becomes smaller. In the limit
where the set becomes infinite (i.e. continuous case) the probability
becomes zero. In order for the method to work in this set-up, a
relaxation is introduced; we relax the acceptance criterion by letting
$\yb_{i}$ lie in a larger set of points i.e.
$\yb_{i} \in \region(\data), \epsilon > 0$. The region can be
defined as $\region (\data) := \{\yb: d(\yb, \data) < \epsilon \}$
where $d(\cdot, \cdot)$ can represent any valid distance. With this
modification, the maintained samples follow an approximate posterior,

\begin{equation} \label{eq:approx_posterior}
  p_{d,\epsilon}(\thetab|\data) \propto Pr(\yb \in
  \region(\data)) p(\thetab)
\end{equation}

\noindent
This method is called \textit{Rejection ABC}.

\subsubsection{Summary Statistics}

When the dimensionality of $\yb \in \mathbb{R}^D$ is high, generating
samples inside $\region (\data)$ becomes rare even with large
acceptance region; this is the curse of dimensionality. As a
representative example if (a) $d$ is set to be the euclidean distance,
then $\region(\data) := \{ \yb: ||\yb - \data||_2^2 < \epsilon^2 \}$
is a hyper-sphere with radius $\epsilon$ and (b) if the prior
$p(\thetab)$ is a uniform distribution in a hyper-cube with side of
length $2\epsilon$, then the probability of drawing a sample inside
the hyper-sphere becomes:

\begin{equation}
  Pr(\yb \in \region (\data)) = Pr(\thetab \in \region (\data)) = \frac{V_{hypersphere}}{V_{hypercube}} = \frac{\pi^{D/2}}{D2^{D-1}\Gamma(D/2)} \rightarrow 0, \quad \text{as} \quad D \rightarrow \infty
\end{equation}

\noindent 
We observe that the probality tends to $0$, independently of
$\epsilon$; enlarging $\epsilon$ will not increase the acceptance
rate. This produces the need for a mapping
$T: \mathbb{R}^{D_1} \rightarrow \mathbb{R}^{D_2}$ where $D_1 > D_2$,
for squeezing the dimensionality of the output. This intermediate step
redefines the area as
$\region(\data) := \{\yb: d(T(\yb), T(\data)) < \epsilon \}$. This
dimensionality-reduction step is called \textit{summary statistic}
extraction, since the distance is not measured on the actual outputs,
but on a summarization (i.e. lower-dimension representation) of them.

\subsubsection{Approximations Introduced}

So far, we have introduced some approximations for infering the
posterior as
$p_{d,\epsilon}(\thetab|\data) \propto Pr(\Y_{\thetab} \in
\region(\data))p(\thetab)$ where
$\region(\data) := \{\yb: d(T(\yb), T(\data)) < \epsilon \}$. These
approximations introduce two different types of errors:

\begin{itemize}
\item $\epsilon$ is chosen to be \textit{big enough}, so that enough
  samples are accepted
\item $T$ introduces loss of information, making possible a $\yb$ far
  away from the $\data$ i.e. $\yb: d(\yb,\data)>\epsilon$, to enter
  the acceptance region after the dimensionality reduction
  $d(T(\yb), T(\data) < \epsilon)$
\end{itemize}

\noindent
In the following sections we will not use the summary statistics in
our expressions, for the notation not to clutter. Though, all the
following propositions are valid with the use of summary statistics.
  
\subsubsection{Optimization Monte Carlo (OMC)}

Based on $\region$, we can define two useful entities; an indicator
function and a conditional distribution.

\subsubsection*{Indicator Function}

The indicator function $\indicator{\region(\yb)}(\xb)$ returns 1 if
$\xb \in \region(\yb)$ and 0 otherwise. If $d(\cdot,\cdot)$ is a
formal distance, due to symmetry
$\indicator{\region(\yb)}(\xb) = \indicator{\region(\xb)}(\yb)$.

\begin{gather} \label{eq:indicator} \indicator{\region(\yb)}(\xb)=
  \left\{
    \begin{array}{ll}
      1 & \mbox{if } \xb \in \region(\yb) \\
      0 & \mbox{else } 
    \end{array} \right. \end{gather}

\subsubsection*{Boxcar Kernel}

The boxcar kernel is the conditional distribution:

\begin{gather}\label{eq:boxcar_kernel}
  p_{d,\epsilon}(\yb|\xb) = \left\{
    \begin{array}{ll}
      c  & \mbox{if } d(\yb,\xb) \leq \epsilon \\
      0 & \mbox{else } 
    \end{array}
  \right. \text{where } c = \frac{1}{\int_{ \{ \yb: d(\yb,\xb) <
      \epsilon \} } d\yb}
\end{gather}
%
Observing the boxcar kernel from the view-point of a data generation
process, where the point $\yb$ is generated after $\xb$, we can make
two important notices:

\begin{itemize}
\item given a specific $\xb$, all values
  $\yb: \yb \in B_{d,\epsilon}(\xb)$ have equal probability to be
  generated
\item if a specific $\yb$ value has been generated, all
  $\xb: \xb \in B_{d,\epsilon}(\yb)$ are equally probable to have been
  the initial point that provoked $\yb$ generation
\end{itemize}
%
Finally, we can also observe that the kernel can be defined through
the indicator function:

\begin{equation}
  p_{d,\epsilon}(\yb|\xb) = c \mathbb{1}_{B_{d,\epsilon}(\yb)}(\xb) = c \mathbb{1}_{B_{d,\epsilon}(\xb)}(\yb)
\end{equation}

\subsubsection*{Initial View}

Based on equation~\ref{eq:approx_posterior} and the Boxcar kernel~\ref{eq:boxcar_kernel}, we can approximate the likelihood as:

\begin{gather} \label{eq:primal_view}
  L_{d, \epsilon}(\thetab) =
  \int_{\yb \in B_\epsilon(\data)}p(\yb|\thetab)d\yb =
  \int_{\yb \in \R^D} \indicator{\region(\data)}(\yb)p(\yb|\thetab)d\yb\\
  \approx \frac{1}{N} \sum_i^N \indicator{\region(\data)}(\yb_i),\text{ where }
  \yb_i \sim M_r(\thetab)
\end{gather}
%
This approach is quite intuitive; approximating the likelihood of a
specific $\thetab$ requires sampling from the data generator and count
the fraction of samples that lie inside the area around the
observations.  Nevertheless, for each distinct evaluation of
$L_{d,\epsilon}(\thetab)$, $N$ new samples are needed to be sampled;
this make this approach quite inconvenient from a computational
point-of-view.

\subsubsection*{Alternative View}

For overcoming the disadvantage introduced above, OMC attempts an
alternative approximation. It samples all the nuisance variables once
$\vb_i \sim p(\vb)$ and it converts the random simulator to a
deterministic mapping $M_d(\thetab, \vb_i)$,

\begin{gather} \label{eq:alt_view}
  L_{d, \epsilon}(\thetab) =
  \int_{\yb \in B_\epsilon(\data)}p(\yb|\thetab)d\yb =
  \int_{\yb \in \R^D} \indicator{\region(\data)}(\yb) p(\yb|\thetab)d\yb\\
  = \int_\yb \int_\vb \indicator{\region(\data)}(\yb) p(\yb|\thetab, \vb) p(\vb) d\yb d\vb \\
  = \int_\vb \indicator{\region(\data)}(\yb=M_d(\thetab, \vb)) p(\vb) d\vb \\
  \approx \frac{1}{N} \sum_i^N \indicator{\region (\data)} (\yb_i) \\
  \text{ where } \yb_i = M_d(\thetab, \vb_i), \vb_i \sim p(\vb)
\end{gather}
%
Based on this approach, the unnormalized approximate posterior can be
defined as:

\begin{equation} \label{eq:aprox_posterior}
  p_{d,\epsilon}(\thetab|\data)
  \propto p(\thetab) \sum_i^N \indicator{ \region(\data)} (\yb_i)
\end{equation}
%
Forming an analogy with the previous approach, we sample many nuisance
variables in order to absorb the randomness of the generator and we
count the fraction of times the deterministic generator produce
outputs close to the observed data. Though it is conceptually similar
to the previous approach, we now overcome the previous disadvantage;
we sample the nuisance variables once (training part) and afterwards
evaluate every $\theta$ based on the predefined expression (inference
part).
