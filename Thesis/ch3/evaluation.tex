The ROMC implementation provides two functions for evaluating the inference results,

\begin{itemize}[label=(\roman*)]
\item \mintinline{python}{romc.compute_divergence(gt_posterior, bounds=None, step=0.1,}

      \mintinline{python}{                        distance="Jensen-Shannon")}

\item \mintinline{python}{romc.compute_ess()}
\end{itemize}

\subsubsection*{Function (i): Compute the divergence between ROMC approximation and a ground-truth posterior}

\pinline{romc.compute_divergence(gt_posterior, bounds=None, step=0.1,}

\pinline{                     distance="Jensen-Shannon")}
\vspace{5mm}

\noindent
This function computes the divergence between the ROMC approximation
and the ground truth posterior. Since the computation is performed
using the Riemann's approximation, this method can only work in low
dimensional parameter spaces; it is suggested to be used for up to the
three-dimensional parameter space. As mentioned in the beginning of
this chapter, in a real-case scenario it is not expected the
ground-truth posterior to be available. However, in cases where the
posterior can be approximated decently well with a numerical approach
(as in the current example) or with some other inference approach,
this function can provide a numerical measure of the aggreement
between the two approximations. The argument \pinline{step} defines
the step used in the Riemann's approximation and the argument
\pinline{distance} can take either the \pinline{Jensen-Shannon} or the
\pinline{KL-divergence} value, for computing the appropriate distance.

\subsubsection*{Function (ii): Compute the effective sample size of the weighted samples}

\mintinline{python}{romc.compute_ess()}
\vspace{5mm}

\noindent
This function computes the Effective Sample Size (ESS) using the
following expression~\autocite{Sudman1967},

\begin{equation} \label{eq:ESS}
  ESS = \frac{(\sum_i w_i)^2}{\sum_i w_i^2}
\end{equation}

The ESS is a usefull measure of the \textbf{actual} sample size, when
the samples are weighted. For example if in a population of $100$
samples one has a very large weight (e.g.\ $\approx 100$) whereas the
rest have small (i.e.\ $\approx 1$), the real sample size is close to
1; one sample dominates over the rest. Hence, the ESS provides a
measure of the equivalent uniformly weighted sample population.

\begin{pythoncode}
res = romc.compute_divergence(wrapper, distance="Jensen-Shannon")                                 
print("Jensen-Shannon divergence: %.3f" % res)
# Jensen-Shannon divergence: 0.025

print("Nof Samples: %d, ESS: %.3f" % (len(romc.result.weights), romc.compute_ess()))
# Nof Samples: 19950, ESS: 16694.816
\end{pythoncode}