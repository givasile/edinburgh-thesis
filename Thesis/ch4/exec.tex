In this section, we will present the execution time of the basic ROMC
functionalities. Apart from performing the inference accurately, one
of the notable advantages of ROMC is its efficiency. In terms of
performance, ROMC holds two key advantages. Firstly, all its subparts
are parallelisable; optimising the objective functions, constructing
the bounding boxes, fitting the local surrogate models, sampling and
evaluating the posterior can be executed in a fully-parallel
fashion. Therefore, the speed-up can be extended as much as the
computational resources allow. Specifically, at the CPU level, the
parallelisation can be incorporate all available cores. A similar
design can be utilised in the case of having a cluster of PCs
available. Parallelising the process at the GPU level is not that
trivial, since the parallel tasks are more complicated than simple
floating-point operations. The second significant advantage concerns
the execution of the training and the inference phase in distinct
timeslots. Therefore, one can consume a lot of training time and
resources but ask for accelerated inference. The use of a lightweight
surrogate model around the optimal point exploits this essential
characteristic; it trades off some additional computational burden at
the training phase, for exercising faster inference later. Since our
current implementation does not support parallel execution
so-far\footnote{The design of our code permits adding this feature in
  a future update.}, in figures \ref{fig:exec_solve},
\ref{fig:exec_regions}, \ref{fig:exec_posterior},
\ref{fig:exec_sample} we can only observe the second advantage. The
example measured in these figures is the simple 1D example, used in
the previous chapter. We observe that fitting local surrogate models
slows down the training phase (estimating the regions) by a factor of
$1.5$, but provides a speed-up of $15$ at the inference phase
i.e.\ approximating unnormalised posterior. This outcome would be
even more potent in larger models, where running the simulator is even
more expensive.



\begin{figure}[h]
    \begin{center}
      \includegraphics[width=0.48\textwidth]{./Thesis/images/chapter4/exec_solve_grad.png}
      \includegraphics[width=0.48\textwidth]{./Thesis/images/chapter4/exec_solve_grad_fit.png}\\
      \includegraphics[width=0.48\textwidth]{./Thesis/images/chapter4/exec_solve_bo.png}
      \includegraphics[width=0.48\textwidth]{./Thesis/images/chapter4/exec_solve_bo_fit.png}          \end{center}
    \caption{Execution time for defining and solving the optimisation
      problems. We observe an increase by a factor of $75$ when switching
      to Bayesian optimisation scheme.}
  \label{fig:exec_solve}
\end{figure}


\begin{figure}[h]
    \begin{center}
      \includegraphics[width=0.48\textwidth]{./Thesis/images/chapter4/exec_regions_grad.png}
      \includegraphics[width=0.48\textwidth]{./Thesis/images/chapter4/exec_regions_grad_fit.png}\\
      \includegraphics[width=0.48\textwidth]{./Thesis/images/chapter4/exec_regions_bo.png}
      \includegraphics[width=0.48\textwidth]{./Thesis/images/chapter4/exec_regions_bo_fit.png}
    \end{center}
    \caption{Execution time for constructing the n-dimensional
      bounding box region and, optionally, fitting the local surrogate
      models. We observe that fitting the surrogate models incurs a
      small increase by a factor of $1.5$.}
  \label{fig:exec_regions}
\end{figure}


\begin{figure}[h]
    \begin{center}
      \includegraphics[width=0.48\textwidth]{./Thesis/images/chapter4/exec_posterior_grad.png}
      \includegraphics[width=0.48\textwidth]{./Thesis/images/chapter4/exec_posterior_grad_fit.png}\\
      \includegraphics[width=0.48\textwidth]{./Thesis/images/chapter4/exec_posterior_bo.png}
      \includegraphics[width=0.48\textwidth]{./Thesis/images/chapter4/exec_posterior_bo_fit.png}
    \end{center}
    \caption{Execution time for evaluating the unnormalised posterior
      approximation. We observe that there is a major speed-up in the
      models with fitted local surogate models, by about a factor $15$.}
  \label{fig:exec_posterior}
\end{figure}


\begin{figure}[h]
    \begin{center}
      \includegraphics[width=0.48\textwidth]{./Thesis/images/chapter4/exec_sample_grad.png}
      \includegraphics[width=0.48\textwidth]{./Thesis/images/chapter4/exec_sample_grad_fit.png}\\
      \includegraphics[width=0.48\textwidth]{./Thesis/images/chapter4/exec_sample_bo.png}
      \includegraphics[width=0.48\textwidth]{./Thesis/images/chapter4/exec_sample_bo_fit.png}
    \end{center}
    \caption{Execution time for sampling from the approximate
      posterior. We observe a small speed-up when the local surogate
      models are fitted, by about a factor $1.5$.}
  \label{fig:exec_sample}
\end{figure}
